\documentclass[12pt,fleqn,answers]{exam}
\usepackage{pifont}
\usepackage{dingbat}
\usepackage{amsmath,amssymb}
\usepackage{epsfig}
\usepackage[colorlinks=true,linkcolor=black,anchorcolor=black,citecolor=black,filecolor=black,menucolor=black,runcolor=black,urlcolor=black]{hyperref}
\usepackage[letterpaper, margin=0.75in]{geometry}
\addpoints
\boxedpoints
\pointsinmargin
\pointname{pts}

\usepackage[thinc]{esdiff}
\usepackage{pdfpages}
\usepackage[final]{microtype}
\usepackage[american]{babel}
\usepackage[T1]{fontenc}
\usepackage{fourier}
\usepackage{isomath}
\usepackage{upgreek,amsmath}
\usepackage{amssymb}

\usepackage{color}
\usepackage{tcolorbox}


\newcommand{\reals}{\mathbf{R}}
\newcommand{\lub}{\mathrm{lub}} 
\newcommand{\glb}{\mathrm{glb}} 
\newcommand{\complex}{\mathbf{C}}
\newcommand{\dom}{\mbox{dom}}
\newcommand{\range}{\mbox{range}}
\newcommand{\integers}{\mathbf{Z}}

\newcommand{\imag}{\, \mathrm{i}}
\newcommand{\fl}{\mathrm{fl}}
\usepackage{xcolor}
\shadedsolutions
\definecolor{SolutionColor}{rgb}{0.95,0.95,0.95}

\usepackage{graphicx}
\newcommand\AM{{\sc am}}
\newcommand\PM{{\sc pm}}
     
\newcommand{\RNum}[1]{\uppercase\expandafter{\romannumeral #1\relax}}

\newcommand{\quiz}{\RNum{3} }
\newcommand{\term}{Spring}
\newcommand{\due}{15:00 \PM}
\newcommand{\class}{CYBR 304 \& MATH 420}

\usepackage {tikz}
\usetikzlibrary {positioning}
\definecolor {processblue}{cmyk}{0.96,0,0,0}

\usepackage{color}  
\usepackage{tikz}  
  
\usetikzlibrary{shapes, arrows, calc, arrows.meta, fit, positioning} 
\tikzset{  
    -Latex,auto,node distance =1.5 cm and 1.3 cm, thick,
    state/.style ={ellipse, draw, minimum width = 0.9 cm}, 
    point/.style = {circle, draw, inner sep=0.18cm, fill, node contents={}},  
    bidirected/.style={Latex-Latex,dashed}, 
    el/.style = {inner sep=2.5pt, align=right, sloped}  
}  

\begin{document}
\large
\noindent\makebox[3.0truein][l]{\textbf{\class}}
\textbf{Name:} \hrulefill \\
\noindent \makebox[3.0truein][l]{\textbf{Exam \quiz}}
\textbf{Row and Seat}:\hrulefill\\

\noindent  Exam \quiz  has questions 1 through  \numquestions \/ with a total of  \numpoints\/  points.   

\vspace{0.1in}
\noindent \begin{tcolorbox}
    \begin{minipage}{6.5in}
    \begin{enumerate}
    
    \normalsize 
    \item \textbf{Show all of your work.} Do not expect to earn full credit for a correct answer without the needed work.
    
    \item Divine intervention is \emph{not} a substitute for showing your work.
    
    \item Respect equality.  Do not write equal ($=$) between expressions that are not equal.
    
           
    \item If your answer is wrong, but your work shows me that you know the major steps in solving a problem, you will likely earn
    some partial credit.
    
    \item Your work should convince me that not only could you correctly solve the given 
    problem, but you could also solve any related problem.
    
    \item If a question asks for an explanation, write your answer as an English sentence. 
    
    \item No talking, no sharing calculators, and no scratch paper.
    \item  Turn your phone off and put it out of sight.
    \item  Clear your desk of everything, except a pencil, an eraser, and a calculator.
    \item If you never make a mistake, you may use ink; otherwise, use a pencil.
    \item Do not unstaple the pages of your exam.
    \item We'll all start at the same time; it is  the polite thing to do.
    \item Write your answers in the space provided. 
    \item If you do not want something graded, erase it or clearly cross it out. 
    \item You may stare at your feet, your paper, or the ceiling, but nowhere else. 
    \item If you wear a  baseball cap, wear it backwards, so I can see your eyes.
    \item Work each problem correctly.
    \item When you are finished, collect your things, place your exam paper in the folder on the front desk,
    and quietly leave the room.
    \item After you turn in your paper, I will not answer questions about the test until after it is graded.
    
    \item Not knowing the rules is not a valid excuse for not following them.
    
    \item Read all directions and problems carefully. 
    \end{enumerate} 
    \end{minipage} 
    
    \end{tcolorbox}

    \newpage



\begin{questions}

\question Find a simple representation of each sum:

\begin{parts}
\part [5]  $\displaystyle \sum_{k=-10}^{10}   \delta_{k,5}$.
\begin{solution}%[3.5in]
We have
\begin{equation*}
\sum_{k=-10}^{10}   \delta_{k,5} = \delta_{-10,5} + \delta_{-9,5} + \cdots \delta_{5,5} + \delta_{6,5} + \cdots
+ \delta_{10,5} = 0 + 0+ \cdots + 1  + 0 + 0 + \dots 0 = 1.
\end{equation*}
Since the Kronecker delta $\delta_{k,5}$ is 1 if $k = 5$ and 0 otherwise, the sum is simply $\delta_{5,5} = 1$.
\end{solution}
\part [5]  $\displaystyle \sum_{k=-10}^{10}   \delta_{k,15}$.
\begin{solution}[3.5in]
We have
\begin{equation*}
\sum_{k=-10}^{10}   \delta_{k,15} = \delta_{-10,15} + \delta_{-9,15} + \cdots + 
\delta_{10,15}  = 0 + 0 \cdots + 0 + 0 = 0.
\end{equation*}
Since the Kronecker delta $\delta_{k,15}$ is 1 if $k = 15$ and 0 otherwise, the summand simplifies to 0, makin
the sum zero.
\end{solution}

\end{parts}

%\newpage

\question Let $f,g$, and $h$ be functions. For some inner product, suppose
\begin{equation*}
\begin{bmatrix}
\langle f,f \rangle & \langle f,g \rangle & \langle f,h \rangle \\
\langle g,f \rangle & \langle g,g \rangle & \langle g,h \rangle \\
\langle h,f \rangle & \langle h,g \rangle & \langle h,h \rangle 
\end{bmatrix} = 
\begin{bmatrix} 2 & 3 & 4 \\
                        3 & 2 & 8 \\
                        4 & 8 & 1
\end{bmatrix}.
\end{equation*}
\begin{parts}
\part [5] Find the numerical value of $\langle f, g + h \rangle $.
\begin{solution}[2.5in]
Using linearity of the inner product, 
\[\langle f, g + h \rangle = \langle f, g \rangle + \langle f, h \rangle = 3 + 4 = 7.\]
\end{solution}
  
\part [5] Find a positive real number $\alpha$ such that  $\langle \alpha f, \alpha f \rangle = 1 $.
\begin{solution}[2.5in]
We need $\langle \alpha f, \alpha f \rangle = \alpha^2 \langle f, f \rangle = 2 \alpha^2  = 1$. Solving for $\alpha$
and keeping in mind that we need a positive solution, we get $\alpha = \frac{1}{\sqrt{2}}$.
\end{solution}
\part [5] Find a real number $\alpha$ such that $f$ and $f + \alpha g$ are perpendicular.
\begin{solution}[2.5in]
We need $\langle f, f + \alpha g \rangle = 0$. Using linearity of the inner product, 
\[0 = \langle f, f + \alpha g \rangle = \langle f, f \rangle + \alpha \langle f, g \rangle = 2 + 3\alpha.\]
Thus, $\alpha = -\frac{2}{3}$.
\end{solution}
\end{parts}

\question [5] Define $P_1(x)  = x$ and $P_2(x)  = 3 x^2 - 1$. Show that $P_1$ and $P_2$ are
perpendicular with respect to the inner product $\langle f, g \rangle = \int_{-1}^1 f(x) g(x) \, \mathrm{d} x$.
\begin{solution}[3.5in]
To show that $P_1$ and $P_2$ are perpendicular, we need to show that $\langle P_1, P_2 \rangle = 0$. Using the given inner product,
\[\langle P_1, P_2 \rangle = \int_{-1}^1 x(3x^2 - 1) \, \mathrm{d} x = \left[ x^2 - \frac{x^4}{2} \right]_{-1}^1 = 0.\]
Thus, $P_1$ and $P_2$ are perpendicular.
\end{solution}

\question [5]  Let $F \in C_{[0,2 \uppi]}$. For $k \in \integers$, define the numbers $c_k$ as
\[c_k = \frac{1}{2 \uppi} \int_0^{2 \uppi} \mathrm{e}^{-\mathrm{i} k x} F(x) \, \mathrm{d} x.\]
Using an n-panel right point rule, find approximate values for the numbers $c_k$ in terms of summation notation.
\begin{solution}[3.5in]
The right point rule for an interval $[a,b]$ with $n$ panels and function $F(x)$ is given by:
\[\int_a^b F(x) \, \mathrm{d} x \approx \sum_{k=1}^n F(x_k) \Delta x,\]
where $x_k = a + k \Delta x$ for $k = 0, 1, \ldots, n$, and $\Delta x = \frac{b-a}{n}$. 
Applying this to our integral, we get:
\[c_k \approx \frac{1}{n} \sum_{j=0}^{n-1} \mathrm{e}^{-\mathrm{i} k x_j} F(x_j) \]
where $x_j = \frac{2\uppi j}{n}$ for $j = 0, 1, \ldots, n-1$.
\end{solution}

\newpage

\question [5] Apply the Gram-Schmidt process to the functions $f_0(x) = 1$ and $f_1(x) = x$ with the inner product
$\langle f,g \rangle = \int_0^1 f(x) g(x) \, \mathrm{d} x$. You should choose $g_0(x) = 1$. Your only task is 
to find a formula for $g_1$.
\begin{solution}[3.5in]
We start with $g_0(x) = f_0(x) = 1$. Then,
\[g_1(x) = f_1(x) - \frac{\langle f_1, g_0 \rangle}{\langle g_0, g_0 \rangle} g_0(x),\]
where 
\[\langle f_1, g_0 \rangle = \int_0^1 x \cdot 1 \, \mathrm{d} x = \frac{1}{2}, \quad \text{and} \quad \langle g_0, g_0 \rangle = \int_0^1 1 \cdot 1 \, \mathrm{d} x = 1.\]
Therefore,
\[g_1(x) = x - \frac{1/2}{1} \cdot 1 = x - \frac{1}{2}.\]
\end{solution}

\question  Consider the family polynomials defined by the recurrence relation:
\[ P_{n+1}(x) = 2 x P_n(x) - P_{n-1}(x), \]
where \( P_0(x) = 1 \) and \( P_1(x) = x \). 

\begin{parts}
\part [5] Using the recurrence relation, find a formula for \( P_2 \).
\begin{solution}[3.5in]
Using the recurrence relation, we have:
\[P_2(x) = 2xP_1(x) - P_0(x) = 2x(x) - 1 = 2x^2 - 1.\]
\end{solution}

\end{parts}

\question[5] Consider the Julia code for evaluation these polynomials

\begin{verbatim}
function P(n::Integer,x::Number)
    if n==0
        1
    elseif n==1 
        x
    else
        2*x*P(n-1,x) - P(n-2,x)
    end
\end{verbatim}

\begin{solution}[3.5in]
The cost to evaluate \verb| 2*x*P(n-1,x) - P(n-2,x) | is the cost 
to evaluate   \verb| P(n-1,x)  | plus the cost to evaluate 
 \verb| P(n-2,x)  | plus three additional floating point operatations ;
 these three additional floating point operations  come from the two explict multiplications in 
 \verb| 2*x*P(n-1,x)  | and from the subtraction in \verb| 2*x*P(n-1,x) - P(n-2,x) |. Thus 
to evaluate $P(n,x)$ using the given Julia code, we can express the number of floating point operations $C_n$ recursively as follows:
\[C_0 = 0, \quad C_1 = 0, \quad \text{and} \quad C_{n+1} = C_n + C_{n-1} + 3 \quad \text{for } n \geq 1.\]
\end{solution}


\end{questions}
\end{document}
