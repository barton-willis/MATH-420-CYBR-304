\documentclass[12pt,fleqn]{exam}
\usepackage{algorithmic,url,calc,amsmath}
\usepackage{amsopn}
\renewcommand{\algorithmicensure}{\textbf{Return:}}
\newenvironment{alphalist}{
  \vspace{-0.4in}
  \begin{enumerate}[(a)]g
    \addtolength{\itemsep}{-1.2\itemsep}}
  {\end{enumerate}}

\newcommand{\integers}{\mathbf{Z}}
\newcommand{\reals}{\mathbf{R}}
\newcommand{\complex}{\mathbf{C}}
\DeclareMathOperator{\rel}{rel}
\DeclareMathOperator{\fl}{fl}
\usepackage{fourier}
\usepackage{color}\shadedsolutions
\definecolor{SolutionColor}{rgb}{1,1,0.7}

\addpoints
\boxedpoints
\pointsinmargin
\pointname{pts}
\begin{document}
\begin{flushleft}


{\bf
  CYBR 304 \\
   Spring \the\year\\
  Review for First Examination
}
\end{flushleft}


\begin{questions}


\question  A floating point number of the type UNK has the form $\pm b_0.b_1 b_2 b_3 b_4 \times 2^e$, where
$b_0$ through $b_4$ are bits and the exponent $e$ is an integer such that $-4 \leq e \leq 3$.  The bit $b_0$ is
not stored--it defaults to 1.

\begin{parts}

\part  Find the \emph{largest} positive number of the type UNK.

\part Find the \emph{smallest positive} number of the type UNK.

\end{parts}

 \question [5] The standard deviation \(\sigma\) of a list of two or more real numbers \(x_1, x_2, \dots, x_n\) is 
\[
   \sigma = \sqrt{\frac{\sum_{k=1}^n (x_k - \bar x)^2}{n}},
\]
where 
\(
 \bar x = \frac{\sum_{k=1}^n x_k }{n}.
\)
An alternative formula for \(\sigma\) is 
\[
   \sigma = \sqrt{\frac{\sum_{k=1}^n x_k^2  -  \left(\sum_{k=1}^n  x_k \right)^2/n}{n}}.
\]
For floating point evaluatation, which formula should you perfer?  Explain.
\begin{solution}%[2.5in]

\end{solution}

%\newpage

 \question[5]  Let \(a\) and \(b\) be  numbers and let \(\oplus\) be floating point addition.  Show that
\[
     \left|  (a  \oplus b)  - (a + b)   \right|  \leq \varepsilon_m  \left ( |a| + |b|  + \left |a + b \right|  \right)  +  \varepsilon_m^2 (| a|  + |b| ).
 \]

\begin{solution}%[3.5in]

\end{solution}


\question[5]  The sequence  \(\displaystyle k \mapsto 8 + \frac{1}{8^k} \) converges to 8. Show that the sequence 
 converges to 5 \emph{linearly}.


\begin{solution}%[3.5in]

\end{solution}



\question Define a sequence \(F\), by \(F_{k+1} = \begin{cases}  1  & \mbox{ if } k =-1  \\  1 - \frac{1}{2} F_k   & \mbox{ otherwise} \\  \end{cases} \).

\begin{parts}

\part[5]  Assuming \(F\) converges,  show that \(F\) converges to \(2/3\).
\begin{solution}%[3.5in]

\end{solution}
\part[5]  Again assuming \(F\) converges,  show that \(F\) converges linearly.
\begin{solution}%[3.5in]

\end{solution}

\end{parts}

%\newpage


\question [5]  Explain how given a term  in the Newton  sequence, how is the next term determined?  Draw a labeled graph that helps explain.
\begin{solution}%[3.5in]

\end{solution}
\question The function \(G(x) = (x^2 - 3 x + 4)/2 \) has a fixed point at 1.  For an intial value that is close to 1, do you expect the fixed point sequence for \(G\) to converge or to diverge?  Why.

\question The function \(H(x) = x^2 - 4 x + 4 \) has a fixed point at 1.  For an intial value that is close to 1, do you expect the fixed point sequence for \(H\) to converge or to diverge?  Why.


\question For real numbers \(x\) and \(y\), you need to compute
\(
     \sqrt{\sqrt{{{y}^{2}}+{{x}^{2}}}+x}.
    \)
An alternative formula for this expression is
\(\frac{\left| y\right| }{\sqrt{\sqrt{{{y}^{2}}+{{x}^{2}}}-x}}\).  

\begin{parts}

\part When \(x < 0\), from the point of view of accuracy, which expression should you use?  Why?

\part When \(x >  0\), from the point of view of accuracy, which expression should you use?  Why?
\end{parts}


\question Understand the function \(\fl\) (See page 36) and its property 
\[
    \left | \fl(x) - x \right|  \leq |x| \varepsilon_m
\] 

\item  Know what we mean by subtractive cancellation.  You should be able to give an example of subtractive cancellation. 

\textbf{Example:}  Suppose \(x = 3.14 \pm 0.01\) and \(y = 3.12 \pm 0.01\).  Thus \(x\) and \(y\) are known accurate to two decimal places. But the difference of these numbers \(x - y\) is a number in the interval \((0, 0.2)\). In \(\pm\) language,
we have \(x - y = 0.1 \pm 0.1\).  Thus although \(x\) and \(y\) are known to be accurate to two decimal places,  the difference \(x - y\) is known accurately to zero digits. This is known as subtractive cancellation.




\question Know the rule that for floating point addition of two floats that \(x  \oplus y = (x + y) (1 + \varepsilon) \), where \(|\varepsilon| \leq \varepsilon_m\).  Using this rule, be able to show that for floats \(a\) and \(b\), we have
\[
     \left|  a  \oplus b - (a + b)   \right|  \leq \varepsilon_m   \left |a + b \right| .
 \]


\question Understand the derivation and the significance of the result
\[
   \left | \frac{x + y -  (\fl(x) + \fl(y))}{x + y}  \right|  \leq \varepsilon_m \frac{|x| + |y|}{|x + y|}.
\]





%\question Know what it means for a function to have a simple zero and a multiple zero (classnotes, page 46, \S1.3).

%\question Know how the secant sequence is defined. Know how to determine its next term from the pervious two terms.

\question Know how the Newton  sequence is defined. Know how to determine its next term from the pervious term.

\question Know what it means for a function to have a fixed point.

\question Know how the fixed point sequence is defined. 

\question Know how to prove that if a fixed point sequence converges, that it converges to a fixed point of the
function.


\question Know what it means for a sequence to converge linearly (page 54).  


\question Know what it means for a sequence to converge quadraticaly  (page 54).  






\question Find the \emph{fixed points} of the function \(x \mapsto {{10\,\left(1-x\right)\,\left(2-x\right)}\over{3}}\).




\question  The sequence  \(\displaystyle k \mapsto 5 + \frac{1}{8^k} \) converges to 5. Show that the sequence 
 converges to 5 \emph{linearly}.






\question Given that a sequence \(F\) converges to 1, what does it mean to say that \(F\) converges to 1 \emph{linearly}?


\question Show that the sequence \(\displaystyle k \mapsto 1 + \left(\frac{1}{2} \right)^k \) converges to 1 \emph{linearly}.




\question  Although \(\displaystyle \sqrt{1+x} - 1 =
\frac{x}{\sqrt{1+x} + 1}\) is an identity, for floating point
evaluation one formula should be
preferred over the other when \(x\) is near zero.
Which formula is better when \(x\) is near zero
and why?
\end{questions}

\end{document}


\question  Although \(\displaystyle \sqrt{1+x} - 1 =
\frac{x}{\sqrt{1+x} + 1}\) is an identity, for floating point
evaluation one formula should be
preferred over the other when \(x\) is near zero.
Which formula is better when \(x\) is near zero
and why?

\vspace{2.0in}





\question  Let \(a\) be a floating point number and let
\(\varepsilon_M\) be the machine epsilon.  Find an upper
bound for
\(
  \left | 1 \oplus a \otimes a  - (1 + a^2)\right|
\).

%\vspace*{2.5in}





\question Draw a nicely labled diagram and show a derivation of the recursive formula for the Newton squence.

\question Know what it means for a sequence to converge linearly.



\end{questions}
\end{document}

