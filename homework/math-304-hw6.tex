\documentclass[12pt,fleqn]{exam}
%\usepackage{pifont}
%\usepackage{dingbat}
\usepackage{amsmath,amssymb}
\usepackage{fleqn}
\usepackage{epsfig}
\usepackage{bbding}
\usepackage{url}
\addpoints
\boxedpoints
\pointsinmargin
\pointname{pts}

\usepackage{xcolor}
\usepackage{framed}
\colorlet{shadecolor}{lightgray!15}
\newenvironment{myproof}
  {\begin{shaded}\begin{proof}}
  {\end{proof}\end{shaded}}
\shadedsolutions
\definecolor{SolutionColor}{rgb}{0.951,0.951,0.951}

\usepackage{amsthm}
\newtheorem{prop}{Proposition}
\usepackage[letterpaper, portrait, margin=0.75in]{geometry}
\usepackage[final]{microtype}
\usepackage[american]{babel}
%\usepackage[T1]{fontenc}
\usepackage{fourier}
\usepackage{isomath}
\usepackage{upgreek,amsmath}
\usepackage{amssymb}
%\usepackage[euler-digits,euler-hat-accent,T1]{eulervm}

\newcommand{\dotprod}{\, {\scriptzcriptztyle
    \stackrel{\bullet}{{}}}\,}

\newcommand{\reals}{\mathbf{R}}
\newcommand{\complex}{\mathbf{C}}
\newcommand{\dom}{\mbox{dom}}
\newcommand{\cover}{{\mathcal C}}
\newcommand{\rat}{\mathbf{Q}}

\newcommand{\curly}[1]{\mathcal #1}
\newcommand{\integers}{\mathbf{Z}}
\newcommand{\vi}{\, \mathbf{i}}
\newcommand{\vj}{\, \mathbf{j}}
\newcommand{\vk}{\, \mathbf{k}}
\newcommand{\bi}{\, \mathbf{i}}
\newcommand{\bj}{\, \mathbf{j}}
\newcommand{\bk}{\, \mathbf{k}}
\DeclareMathOperator{\Arg}{\mathrm{Arg}}
\DeclareMathOperator{\Ln}{\mathrm{Ln}}
\newcommand{\imag}{\, \mathrm{i}}
\newcommand{\range}{\mathrm{range}}
\newcommand{\true}{\mathrm{True}}
\newcommand{\saw}{\mathrm{saw}}
\newcommand{\fl}{\mathrm{Fl}}
\newcommand{\mach}{\varepsilon_m}
\newcommand{\HM}{\mathrm{HM}}
\newcommand{\Inf}{\mathrm{Inf}}
\newcommand{\sigmoid}{\mathrm{sigmoid}}
\newcommand{\erf}{\mathrm{erf}}

\usepackage{enumerate}
\usepackage{verbatim}
\newenvironment{alphalist}{
  \begin{enumerate}[(a)]
    \addtolength{\itemsep}{-0.5\itemsep}}
  {\end{enumerate}}
 
\usepackage{amsthm}
\newtheorem{Rubbish}{Theorem}
\usepackage{graphicx}

%\usepackage{tgschola} %to look retro
\newenvironment{mypar}[2]
  {\begin{list}{}%
    {\setlength\leftmargin{#1}
    \setlength\rightmargin{#2}}
    \item[]}
  {\end{list}}
  
\newcommand{\quiz}{6}
\newcommand{\term}{Spring}
\usepackage{units}
\usepackage{xspace}
\makeatletter
\DeclareRobustCommand{\maybefakesc}[1]{%
  \ifnum\pdfstrcmp{\f@series}{\bfdefault}=\z@
    {\fontsize{\dimexpr0.8\dimexpr\f@size pt\relax}{0}\selectfont\uppercase{#1}}%
  \else
    \textsc{#1}%
  \fi
}
\newcommand\AM{\,\maybefakesc{am}\xspace}
\newcommand\PM{\,\maybefakesc{pm}\xspace}


\usepackage{fancyvrb,newverbs,xcolor}
\usepackage{lipsum}% just for this example

\definecolor{cverbbg}{gray}{0.93}

\newenvironment{cverbatim}
 {\SaveVerbatim{cverb}}
 {\endSaveVerbatim
  \flushleft\fboxrule=0pt\fboxsep=.5em
  \colorbox{cverbbg}{\BUseVerbatim{cverb}}%
  \endflushleft
}
\newenvironment{lcverbatim}
 {\SaveVerbatim{cverb}}
 {\endSaveVerbatim
  \flushleft\fboxrule=0pt\fboxsep=.5em
  \colorbox{cverbbg}{%
    \makebox[\dimexpr\linewidth-2\fboxsep][l]{\BUseVerbatim{cverb}}%
  }
  \endflushleft
}

\newcommand{\ctexttt}[1]{\colorbox{cverbbg}{\texttt{#1}}}
\newverbcommand{\cverb}
  {\setbox\verbbox\hbox\bgroup}
  {\egroup\colorbox{cverbbg}{\box\verbbox}}
\begin{document}
\vspace{0.1in}
\noindent\makebox[3.0truein][l]{ \textbf{CYRR 304}}
{\bf Name:}  \\
\noindent \makebox[3.0truein][l]{\textbf{Homework \quiz, \term \/ \the\year}}
%{\bf Row:}\hrulefill\
\vspace{0.1in}

\begin{quote}
\noindent \emph{“To pay attention, this is our endless and proper work.”} \\ $\phantom{xxx} $ \hfill {\sc Mary Oliver}
\end{quote}

\noindent  Homework \quiz\/  has questions 1 through  \numquestions \/ with a total of of  \numpoints\/  points. 
Your recorded score will be scaled to twenty points. The point
value for each question or part of a question is in the box following each question or part of a question.
This work is due \textbf{Saturday 9 March  } at 11:59 \PM.  

For this assignment, convert your Jupyter notebook (a IPYNB file)  to HTML and submit the HTML file to Canvas.

The sigmoid function is frequently used in machine learning, but oddly, there is not a single agreed definition of this function.
For this assignment, we'll define the sigmoid function by
\begin{equation}
  \sigmoid(x) = \frac{2}{\sqrt{\uppi}} \int_0^x \mathrm{e}^{-t} \, \mathrm{d} t.
\end{equation}
But the standard name of this function is $\erf$, not $\sigmoid$.  The function $\erf$ is important in statistics.\
Oftentimes in machine learning, it's not important that the sigmoid function be especially accurate, but it is 
important for it to be super fast.

\begin{questions}

\question [10] Use Gadfly to graph the function $\erf$ on the interval $[-1,1]$.  To evaluate $\erf$ using Julia, you 
will need to use the package manager to install the package \texttt{SpecialFunctions}. Based on the graph, does 
the function $\erf$ appear to be and even or odd function?

\question [10] Find the third degree polynomial that interpolates the $\erf$ function. For the interpolation points,
use $0,1/3, 2/3, 1$.

\question [10] Write a Julia function $\sigmoid$ that evaluates the third degree polynomial that you found in the previous 
question. Use the odd property of $\sigmoid$ to extend your function to accurately evaluate $\sigmoid(x)$ for 
$x \in [-1,0]$. 

\question [10] Use Gadfly to graph $y = |\erf(x) - \sigmoid(x)|$. Use the graph to visually determine 
\begin{equation}
   \max\limits_{x \in [-1,1]}  |\erf(x) - \sigmoid(x)|.
\end{equation}

\question [10] Use @btime to compare the time to evaluate $\erf(0.6)$ and $\sigmoid(0.6)$. Is your function faster?
If not, try again. Likely, if you defined your function with the stab operator \texttt{->}, it will be slow. Try defining it without
the stab operator.  To use @btime, you will need to use the package manager to install BenchmarkTools

\question [10] Find the third degree polynomial that interpolates the $\erf$ function, but this time for the interpolation points,
use $\frac{1}{2} + \frac{1}{2} \cos \left (\frac{2 k -1}{8} \uppi \right)$ for $k \in \{1,2,3,4\}$. Write a Julia function $\sigmoid2$ that evaluates the third degree polynomial that you found in this problem. Again, use the odd property of $\sigmoid$ to extend your function to accurately evaluate $\sigmoid(x)$ for 
$x \in [-1,0]$. Is this function more accurate than the version that uses equally spaced interpolation points? Explain.



\end{questions}

\end{document}
