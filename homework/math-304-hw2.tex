\documentclass[12pt,fleqn]{exam}
\usepackage{pifont}
\usepackage{dingbat}
\usepackage{amsmath,amssymb}
\usepackage{fleqn}
\usepackage{epsfig}
%\usepackage{mathptm}
%\usepackage{euler}
\usepackage{bbding}
\usepackage{url}
\addpoints
\boxedpoints
\pointsinmargin
\pointname{pts}

\usepackage{xcolor}
\usepackage{framed}
\colorlet{shadecolor}{lightgray!15}
\newenvironment{myproof}
  {\begin{shaded}\begin{proof}}
  {\end{proof}\end{shaded}}
\shadedsolutions
\definecolor{SolutionColor}{rgb}{0.951,0.951,0.951}

\usepackage{amsthm}
\newtheorem{prop}{Proposition}
\usepackage[letterpaper, portrait, margin=0.75in]{geometry}
\usepackage[final]{microtype}
\usepackage[american]{babel}
%\usepackage[T1]{fontenc}
\usepackage{fourier}
\usepackage{isomath}
\usepackage{upgreek,amsmath}
\usepackage{amssymb}
%\usepackage[euler-digits,euler-hat-accent,T1]{eulervm}

\newcommand{\dotprod}{\, {\scriptzcriptztyle
    \stackrel{\bullet}{{}}}\,}

\newcommand{\reals}{\mathbf{R}}
\newcommand{\complex}{\mathbf{C}}
\newcommand{\dom}{\mbox{dom}}
\newcommand{\cover}{{\mathcal C}}
\newcommand{\rat}{\mathbf{Q}}

\newcommand{\curly}[1]{\mathcal #1}
\newcommand{\integers}{\mathbf{Z}}
\newcommand{\vi}{\, \mathbf{i}}
\newcommand{\vj}{\, \mathbf{j}}
\newcommand{\vk}{\, \mathbf{k}}
\newcommand{\bi}{\, \mathbf{i}}
\newcommand{\bj}{\, \mathbf{j}}
\newcommand{\bk}{\, \mathbf{k}}
\DeclareMathOperator{\Arg}{\mathrm{Arg}}
\DeclareMathOperator{\Ln}{\mathrm{Ln}}
\newcommand{\imag}{\, \mathrm{i}}
\newcommand{\range}{\mathrm{range}}
\newcommand{\true}{\mathrm{True}}
\newcommand{\saw}{\mathrm{saw}}
\newcommand{\fl}{\mathrm{Fl}}
\newcommand{\mach}{\varepsilon_m}
\newcommand{\HM}{\mathrm{HM}}
\newcommand{\Inf}{\mathrm{Inf}}
\usepackage{enumerate}
\usepackage{verbatim}
\newenvironment{alphalist}{
  \begin{enumerate}[(a)]
    \addtolength{\itemsep}{-0.5\itemsep}}
  {\end{enumerate}}
 
\usepackage{amsthm}
\newtheorem{Rubbish}{Theorem}
\usepackage{graphicx}

%\usepackage{tgschola} %to look retro
\newenvironment{mypar}[2]
  {\begin{list}{}%
    {\setlength\leftmargin{#1}
    \setlength\rightmargin{#2}}
    \item[]}
  {\end{list}}
  
\newcommand{\quiz}{2}
\newcommand{\term}{Spring}
\usepackage{units}
\usepackage{xspace}
\makeatletter
\DeclareRobustCommand{\maybefakesc}[1]{%
  \ifnum\pdfstrcmp{\f@series}{\bfdefault}=\z@
    {\fontsize{\dimexpr0.8\dimexpr\f@size pt\relax}{0}\selectfont\uppercase{#1}}%
  \else
    \textsc{#1}%
  \fi
}
\newcommand\AM{\,\maybefakesc{am}\xspace}
\newcommand\PM{\,\maybefakesc{pm}\xspace}

\begin{document}
\large
\vspace{0.1in}
\noindent\makebox[3.0truein][l]{ \textbf{CYRR 304}}
{\bf Name:}  \\
\noindent \makebox[3.0truein][l]{\textbf{Homework \quiz, \term \/ \the\year}}
%{\bf Row:}\hrulefill\
\vspace{0.1in}

\begin{quote}
“\emph{Facts are stubborn things, but statistics are pliable.}”  \hfill {\sc Mark Twain} 

\end{quote}
\noindent  Homework \quiz\/  has questions 1 through  \numquestions \/ with a total of of  \numpoints\/  points. 
Your recorded score will be scaled to twenty points. The point
value for each question or part of a question is in the box following each question or part of a question.
This work is due \textbf{Saturday 3 Feb  } at 11:59 \PM. For this assignment, turn in a Jupyter notebook (a IPYNB file) to Canvas


\vspace{0.1in}
The harmonic mean $\HM$ of positive numbers $x_1, x_2, \cdots, x_n$ is defined as
\begin{equation*}
  \HM(x_1, x_2, \cdots, x_n) = \frac{n}{\sum_{k=1}^n \frac{1}{x_k}}.
\end{equation*}
For example
\begin{equation*}
  \HM(5,45) = \frac{2}{\frac{1}{5} + \frac{1}{45}} = 9.
\end{equation*}
The harmonic mean has lots of applications to statistics, engineering, and finance.   One simple example is that if
you drive from Kearney to North Platte at 60 mph and turn around and drive back to Kearney at 70 mph, your average speed for the trip is the harmonic, not arithmetic mean of 60 and 70; thus your average speed is 
\begin{equation*}
\frac{2}{\frac{1}{60} + \frac{1}{70}} = \frac{840}{13} \approx 64.6 \mathrm{mph}.
\end{equation*}

The harmonic mean has the property that for all $\alpha \in \reals_{\neq 0}$ and for all $x_1, x_2, \cdots x_n \in \reals_{> 0}$,
the equation
\begin{equation*}
  \HM(x_1, x_2, \cdots x_n) =\alpha  \HM(x_1 / \alpha , x_2 /\alpha , \cdots, x_n/\alpha ).
\end{equation*}
is an identity. Additionally, for all $x \in \reals_{>0}$, the equation
$  \HM(x,x, \cdots ,x ) =x$ is an identity.

The Julia package   StatsBase has a function `harmmean' that computes the harmonic mean of the members
of a vector.  You'll need to download and install this package. This is done using the package manager

\begin{verbatim}
julia> using Pkg
julia> Pkg.add("StatsBase")
   Resolving package versions...
 \end{verbatim}
Once you have downloaded and installed the this package, you don't need to do these steps again. To make the
package available enter

\begin{verbatim}
julia> using StatsBase;

julia> x = 2.0^-1074
5.0e-324

julia> harmmean([x,x])
0.0
\end{verbatim}
Given the identity $  \HM(x,x) =x$, a better output is 5.0e-324. But it's not mysterious why the output is 0.
The reciprocal of the denormalized number $2.0^{-1074}$ overflows to Inf. But in Julia $1/\Inf$ is zero. So effectively Julia (miss) computes 
\begin{equation} 
  2 / (1/{5.0\times 10^{-324}}  + 1/{5.0\times 10^{-324}}) = 2 /(\Inf + \Inf) = 2/\Inf =0.
 \end{equation}
Here is a proposed function `harmonicMean` that will better handle denormalized inputs
\begin{verbatim}
function harmonicMean(L)
   n = length(L)
   m = max(map(abs, L)...) # maximum abs value
   L = map(x -> m/x,L)  
   m*(n / sum(L))
end
\end{verbatim}


\begin{questions}  
\question[1] For $x = 2.0^{-1074}$, find the value of `harmonicMean([x,x])'  and `harmonicMean([x,x,x])'. Does this
version of the harmonic mean seem to resolve the issue for the function from the StatsBase package?


\question[1] In Julia arithmetic $1/\Inf = 0$.  Using this standard, we should have
  \begin{equation*}
  \HM(\infty, 5,45) = \frac{3}{\frac{1}{\infty} +  \frac{1}{5} + \frac{1}{45} } =
                           \frac{3}{\frac{1}{5} + \frac{1}{45} }
                           = \frac{27}{2}.
\end{equation*}
Evaluate $\mathrm{harmonicMean}([5,45,\Inf])$. The value should be 13.5. Is it?  Step through the Julia code and
explain why $\mathrm{harmonicMean}([5,45,\Inf])$ returns $\mathrm{NaN}$.  \textbf{Hint}  In Julia arithmetic,
we have $\Inf/\Inf = \mathrm{NaN}$.

\question [1] To fix the defect in $\mathrm{harmonicMean}$ when one or more input is $\Inf$, one approach is to
delete every member of the input that is $\Inf$ before computing $m$. To do this, use
\begin{verbatim}
 filter!(x-> x != Inf,L)  # remove Inf from L
\end{verbatim}
The Julia function filter! modifies its argument. Similarly, remove every input of $-\Inf$. Modify  $\mathrm{harmonicMean}$  to do this.

\question [1] Test your modified  $\mathrm{harmonicMean}$  function with the input  
\begin{equation}
 \mathrm{harmonicMean}[2.0^{-1074},2.0^{-1074}, \Inf])
 \end{equation}
  Is the output correct? 

\question [1] Using both $\mathrm{harmonicMean}$ and the function $\mathrm{harmmean}$ from StatBase,
find the harmonic mean of the one thousand member vector $[10,10, 10, \dots, 10]$. To do this use
\begin{verbatim}
   harmonicMean([10.0 for k=1:10^3])
\end{verbatim}
and
\begin{verbatim}
   harmmean([10.0 for k=1:10^3])
\end{verbatim}
Which function is more accurate for this test? Why? Congratulations, you have built a better Julia function for the harmonic mean.

\end{questions}
\end{document}