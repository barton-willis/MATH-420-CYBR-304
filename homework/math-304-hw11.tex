\documentclass[12pt,fleqn]{exam}
\usepackage{amsmath,fleqn, url}
\usepackage{enumerate,color}
%\pagestyle{empty}

\usepackage{graphicx}
\usepackage[letterpaper,margin=1.0in]{geometry}
\usepackage[english]{babel}
\usepackage[final,babel]{microtype}
\usepackage{amsmath, amssymb, amsthm, upgreek, xfrac, isomath}
\frenchspacing
\usepackage[T1]{fontenc}
%\usepackage{lmodern}
\usepackage[upright]{fourier} %options: sloped, upright, widespace
%\usepackage[euler-digits,euler-hat-accent,T1]{eulervm}

\newcommand{\reals}{\mathbf{R}}
\newcommand{\integers}{\mathbf{Z}}
\newcommand{\eu}{\mathrm{e}}
\newcommand{\range}{\mathrm{range}}
\usepackage{algorithm}
\usepackage[noend]{algorithmic}
\usepackage{isomath}
\usepackage{graphicx}
\renewcommand{\algorithmicrequire}{\textbf{input}}
\renewcommand{\algorithmicensure}{\textbf{return}}

\usepackage[T1]{fontenc}


\usepackage[usenames,dvipsnames]{xcolor}


\newcommand{\algo}[1]{\vspace{0.1in}
      \begin{flushleft}
         {\bf Algorithm} {\em #1}
      \end{flushleft}
      \vspace{0.1in}}

\newenvironment{myalgorithm}[1]
  {\setcounter{myalgo}{\value{myalgo}+1} 
   \textbf{Algorithm \themysec.\hspace{-0.15in}\myalgo} {\em #1} \\ \vspace{-0.25in}}
  

\DeclareMathOperator{\fl}{fl}
\newenvironment{alphalist}{
  \begin{enumerate}[(a)]
    \addtolength{\itemsep}{-1.0\itemsep}}
  {\end{enumerate}}

\newcommand{\quiz}{11}
\newcommand{\term}{Spring}
\usepackage{units}
\usepackage{xspace}
\makeatletter
\DeclareRobustCommand{\maybefakesc}[1]{%
  \ifnum\pdfstrcmp{\f@series}{\bfdefault}=\z@
    {\fontsize{\dimexpr0.8\dimexpr\f@size pt\relax}{0}\selectfont\uppercase{#1}}%
  \else
    \textsc{#1}%
  \fi
}
\newcommand\AM{\,\maybefakesc{am}\xspace}
\newcommand\PM{\,\maybefakesc{pm}\xspace}

\addpoints
\boxedpoints
\pointsinmargin
\pointname{pts}
\begin{document}

\vspace{0.1in}
\noindent\makebox[3.0truein][l]{ \textbf{CYRR 304}}
{\bf Name:}  \\
\noindent \makebox[3.0truein][l]{\textbf{Homework \quiz, \term \/ \the\year}}
%{\bf Row:}\hrulefill\
\vspace{0.1in}

\noindent  Homework 11  has questions 1 through  \numquestions \/ with a total of of  \numpoints\/  points. The point
value for each question or part of a question is in the box following each question or part of a question.
This work is due \textbf{Saturday 27 April  } at 11:59 \PM.  

For this assignment, neatly handwrite your work, digitize it, and turn in the digital copy to Canvas.  If you use Julia
to draw the graph for the last problem, convert it to HTML and submit that; if you use Desmos, export it as a png file 
and include the png file.

\begin{questions}

\question A sequence $c$ is defined recursively by
\begin{equation}
  c_n = \begin{cases} 0 & n = 0 \\ 2 c_{n-1} + 1 & n \geq 1 \end{cases}.
\end{equation}

\begin{parts}

\part [10]  Find a formula for the generating function for the sequence $c$. That is, find a formula for $\sum_{k=0}^\infty c_n z^n$. 


\part[10] Use the generating function to find a non recursive formula for the sequence $c$. You might like to use the fact that
\begin{equation}
  \sum_{k=1}^\infty z^k =  \frac{z}{1-z}.
\end{equation}
And you might like to use the fact that
\begin{equation}
    \frac{1}{1- 2 z} \times \frac{z}{1-z} = \frac{1}{z-1} - \frac{1}{2 z - 1}.
\end{equation}
To earn credit for this question, you must show all of your work. 

\part[10] Check that your formula for the sequence $c$ matches with $c_0 = 0, c_1 = 1, c_2 = 3, c_3 = 7, c_4 = 15,
c_5 = 31$.


\end{parts}

\question Define functions $f_1(x) = \frac{1}{\sqrt{\uppi}} \sin(x)$, $f_2(x) = \frac{1}{\sqrt{\uppi}} \sin(2x)$, 
$f_3(x) = \frac{1}{\sqrt{\uppi}} \sin(3 x)$,  $f_4(x) = \frac{1}{\sqrt{\uppi}} \sin(4 x)$, $\cdots, $  $f_{10}(x) = \frac{1}{\sqrt{\uppi}} \sin(10 x)$. Then the set of functions $\{f_0, f_1, \dots, f_{10}\}$ is orthonormal.  

\begin{parts}

\part [10] Find numbers
$c_1, c_2, \dots, c_{10}$ that minimize the function
\begin{equation*}
   (c_1, c_2, \dots, c_{10}) \in \reals^{10} \mapsto  \int_{-\uppi}^{\uppi} \left(x - \sum_{k=1}^{10} \frac{1}{\sqrt{\uppi}} c_k \sin(k x) \right)^2 \, \mathrm{d} x.
\end{equation*}
You might like to use the fact that for all $k \in \integers_{>0}$, we have
\begin{equation}
\int_{\operatorname{-}\ensuremath{\pi} }^{\ensuremath{\pi} }{\left. \frac{x \sin{\left( k x\right) }}{\sqrt{\ensuremath{\pi} }}dx\right.}\operatorname{=}\operatorname{-}\left( \frac{2 \sqrt{\ensuremath{\pi} } {{\left( \operatorname{-}1\right) }^{k}}}{k}\right) 
\end{equation}


\part [10] Using the numbers $c_1$ through $c_{10}$ you found in the previous question, ask a graphing utility
to graph both $y = x$ and $y = \sum_{k=1}^{10} \frac{1}{\sqrt{\uppi}} c_k \sin(k x))$ on the same graph. 

For a graphing utility, you can use Julia and Gadfly, or optionally, you can use Desmos; if you have never
used Desmos, here is a link to a Desmos graph \url{https://www.desmos.com/calculator/kauijqwiri}
\end{parts}

\end{questions}
\end{document}