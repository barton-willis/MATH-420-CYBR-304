\documentclass[12pt,fleqn]{exam}
\usepackage{amsmath,fleqn, url}
\usepackage{enumerate,color}
%\pagestyle{empty}

\usepackage{graphicx}
\usepackage[letterpaper,margin=1.0in]{geometry}
\usepackage[english]{babel}
\usepackage[final,babel]{microtype}
\usepackage{amsmath, amssymb, amsthm, upgreek, xfrac, isomath}
\frenchspacing
\usepackage[T1]{fontenc}
%\usepackage{lmodern}
\usepackage[upright]{fourier} %options: sloped, upright, widespace
%\usepackage[euler-digits,euler-hat-accent,T1]{eulervm}

\newcommand{\reals}{\mathbf{R}}
\newcommand{\integers}{\mathbf{Z}}
\newcommand{\eu}{\mathrm{e}}
\usepackage{algorithm}
\usepackage[noend]{algorithmic}
\usepackage{isomath}
\usepackage{graphicx}
\renewcommand{\algorithmicrequire}{\textbf{input}}
\renewcommand{\algorithmicensure}{\textbf{return}}


\newcommand{\algo}[1]{\vspace{0.1in}
      \begin{flushleft}
         {\bf Algorithm} {\em #1}
      \end{flushleft}
      \vspace{0.1in}}

\newenvironment{myalgorithm}[1]
  {\setcounter{myalgo}{\value{myalgo}+1} 
   \textbf{Algorithm \themysec.\hspace{-0.15in}\myalgo} {\em #1} \\ \vspace{-0.25in}}
  

\DeclareMathOperator{\fl}{fl}
\newenvironment{alphalist}{
  \begin{enumerate}[(a)]
    \addtolength{\itemsep}{-1.0\itemsep}}
  {\end{enumerate}}

\newcommand{\quiz}{8}
\newcommand{\term}{Spring}
\usepackage{units}
\usepackage{xspace}
\makeatletter
\DeclareRobustCommand{\maybefakesc}[1]{%
  \ifnum\pdfstrcmp{\f@series}{\bfdefault}=\z@
    {\fontsize{\dimexpr0.8\dimexpr\f@size pt\relax}{0}\selectfont\uppercase{#1}}%
  \else
    \textsc{#1}%
  \fi
}
\newcommand\AM{\,\maybefakesc{am}\xspace}
\newcommand\PM{\,\maybefakesc{pm}\xspace}

\addpoints
\boxedpoints
\pointsinmargin
\pointname{pts}
\begin{document}

\vspace{0.1in}
\noindent\makebox[3.0truein][l]{ \textbf{CYRR 304}}
{\bf Name:}  \\
\noindent \makebox[3.0truein][l]{\textbf{Homework \quiz, \term \/ \the\year}}
%{\bf Row:}\hrulefill\
\vspace{0.1in}

\begin{quote}
\noindent \emph{
“Instructions for living a life.
Pay attention.
Be astonished.
Tell about it.” \\ 
$\phantom{xxx}$ \hfill {\sc Mary Oliver}}
  
\end{quote}


\noindent  Homework 8  has questions 1 through  \numquestions \/ with a total of of  \numpoints\/  points. 
Your recorded score will be scaled to twenty points. The point
value for each question or part of a question is in the box following each question or part of a question.
This work is due \textbf{Saturday 30 March  } at 11:59 \PM.  
\begin{questions}

\question You need a pretty good value of  $\int_0^{10}  \sin(x^2) \, \mathrm{d} x$. To do this, you will use the $n$ panel trapezoidal rule.

\begin{parts}

\part [5] Use Gadfly to plot a graph of the \emph{second derivative} of $ x \mapsto \sin(x^2)$.  Visually determine the value of \(\underset{x \in [0,10]}{\max}
\left  |2 \cos{\left( {{x}^{2}}\right) }-4 {{x}^{2}} \sin{\left( {{x}^{2}}\right) } \right |\).

\part [5] Find the least value for $n$ needed so that the $n$-panel trapezoidal rule gives an error of no more than $10^{-8}$.

\part[5] Using that value of $n$, estimate the value of  $\int_0^{10}  \sin(x^2) \, \mathrm{d} x$.



\end{parts}


\question  Let $T_n$ be the $n$-panel trapezoidal rule value for $\int_0^1 \sqrt{x} \,  \mathrm{d} x$. 

\begin{parts}

\part [5] Sketch a graph of $T$ on the interval $ 1   \dots 500.$

\part [5] The true value of  $\int_0^1 \sqrt{x} \,  \mathrm{d} x$ is $2/3$.  Sketch a graph of $n \mapsto n^2 (T_n - 2/3)$ on the interval $ 1 \dots  500$. Does it appear that
the graph has a horizontal asymptote? Doesn't the theory say it should?  Explain. (Hint is $x \mapsto \sqrt{x}$ sufficiently smooth on $[0,1]$?)



\end{parts}

\end{questions}
\end{document}