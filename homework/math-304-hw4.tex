\documentclass[12pt,fleqn]{exam}
\usepackage{pifont}
\usepackage{dingbat}
\usepackage{amsmath,amssymb}
\usepackage{fleqn}
\usepackage{epsfig}
%\usepackage{mathptm}
%\usepackage{euler}
\usepackage{bbding}
\usepackage{url}
\addpoints
\boxedpoints
\pointsinmargin
\pointname{pts}

\usepackage{xcolor}
\usepackage{framed}
\colorlet{shadecolor}{lightgray!15}
\newenvironment{myproof}
  {\begin{shaded}\begin{proof}}
  {\end{proof}\end{shaded}}
\shadedsolutions
\definecolor{SolutionColor}{rgb}{0.951,0.951,0.951}

\usepackage{amsthm}
\newtheorem{prop}{Proposition}
\usepackage[letterpaper, portrait, margin=0.75in]{geometry}
\usepackage[final]{microtype}
\usepackage[american]{babel}
%\usepackage[T1]{fontenc}
\usepackage{fourier}
\usepackage{isomath}
\usepackage{upgreek,amsmath}
\usepackage{amssymb}
%\usepackage[euler-digits,euler-hat-accent,T1]{eulervm}

\newcommand{\dotprod}{\, {\scriptzcriptztyle
    \stackrel{\bullet}{{}}}\,}

\newcommand{\reals}{\mathbf{R}}
\newcommand{\complex}{\mathbf{C}}
\newcommand{\dom}{\mbox{dom}}
\newcommand{\cover}{{\mathcal C}}
\newcommand{\rat}{\mathbf{Q}}

\newcommand{\curly}[1]{\mathcal #1}
\newcommand{\integers}{\mathbf{Z}}
\newcommand{\vi}{\, \mathbf{i}}
\newcommand{\vj}{\, \mathbf{j}}
\newcommand{\vk}{\, \mathbf{k}}
\newcommand{\bi}{\, \mathbf{i}}
\newcommand{\bj}{\, \mathbf{j}}
\newcommand{\bk}{\, \mathbf{k}}
\DeclareMathOperator{\Arg}{\mathrm{Arg}}
\DeclareMathOperator{\Ln}{\mathrm{Ln}}
\newcommand{\imag}{\, \mathrm{i}}
\newcommand{\range}{\mathrm{range}}
\newcommand{\true}{\mathrm{True}}
\newcommand{\saw}{\mathrm{saw}}
\newcommand{\fl}{\mathrm{Fl}}
\newcommand{\mach}{\varepsilon_m}
\newcommand{\HM}{\mathrm{HM}}
\newcommand{\Inf}{\mathrm{Inf}}
\usepackage{enumerate}
\usepackage{verbatim}
\newenvironment{alphalist}{
  \begin{enumerate}[(a)]
    \addtolength{\itemsep}{-0.5\itemsep}}
  {\end{enumerate}}
 
\usepackage{amsthm}
\newtheorem{Rubbish}{Theorem}
\usepackage{graphicx}

%\usepackage{tgschola} %to look retro
\newenvironment{mypar}[2]
  {\begin{list}{}%
    {\setlength\leftmargin{#1}
    \setlength\rightmargin{#2}}
    \item[]}
  {\end{list}}
  
\newcommand{\quiz}{3}
\newcommand{\term}{Spring}
\usepackage{units}
\usepackage{xspace}
\makeatletter
\DeclareRobustCommand{\maybefakesc}[1]{%
  \ifnum\pdfstrcmp{\f@series}{\bfdefault}=\z@
    {\fontsize{\dimexpr0.8\dimexpr\f@size pt\relax}{0}\selectfont\uppercase{#1}}%
  \else
    \textsc{#1}%
  \fi
}
\newcommand\AM{\,\maybefakesc{am}\xspace}
\newcommand\PM{\,\maybefakesc{pm}\xspace}

\begin{document}
\vspace{0.1in}
\noindent\makebox[3.0truein][l]{ \textbf{CYRR 304}}
{\bf Name:}  \\
\noindent \makebox[3.0truein][l]{\textbf{Homework \quiz, \term \/ \the\year}}
%{\bf Row:}\hrulefill\
\vspace{0.1in}

\begin{quote}
\emph{“Compound interest is the eighth wonder of the world. He who understands it, earns it; he who doesn't, pays it”.}  \hfill attributed to  {\sc Albert Einstein} 

\end{quote}
\noindent  Homework \quiz\/  has questions 1 through  \numquestions \/ with a total of of  \numpoints\/  points. 
Your recorded score will be scaled to twenty points. The point
value for each question or part of a question is in the box following each question or part of a question.
This work is due \textbf{Saturday 10 Feb  } at 11:59 \PM.  

For this assignment, convert your  Jupyter notebook (a IPYNB file)
 to HTML and submit the HTML file to Canvas. To convert you Jupyter notebook to HTML, do File --> Download as --> HTML. (For the File menu, look toward the 
 upper right corner.)
 
 \begin{questions}
 
 \question This week's assignment is motivated by a problem in financial math. But we'll skip the details of the origin of the problem and skip directly 
 to the numerical analysis.  For various value of the number $q$, we need to solve the equation
 \begin{equation}
   q = z + z^2 + z^3 + z^4
 \end{equation} 
 for $z$.   Let's use fixed point iteration to do do this.  There are lots of ways to convert this equation to fixed point form; here is one way
 \begin{equation}
   z = q  - \left (z^2 + z^3 + z^4\right).
 \end{equation} 
 
 \begin{parts}
 
 \part [5] Suppose $q = 4.35$.  Use Gadfly to graph both $z \mapsto z$ and $z \mapsto  q  - \left (z^2 + z^3 + z^4\right)$ on the same graph.
 Do you think the fixed point sequence will converge?  Explain.
 
 \part [5]   For best convergence of a fixed point sequence for a function $F$, we would like the derivative of $F$ be zero at the fixed point.  And that gives and idea.  Instead of using the fixed point method on the function $F = z \mapsto q - \left(z^4 + z^3 + z^2 + z\right)$, let's find the fixed point
 sequence of $G = z \mapsto F(z) + s (F(z) - z)$, where $s$ is some real number. The functions $F$ and $G$ have the same fixed points--let's choose
 the number $s$ so that $G^\prime(1) = 0$.  Thus $0 = F'(1) + s(F^\prime(1)  -1)$. But $F^\prime(1) = -9$, so we want $s = -\frac{9}{10}$. This choice gives $G = z \mapsto \frac{1}{10} F(z) + \frac{9}{10} z$.
 
 Use Gadfly to graph both $z \mapsto z$ and $z \mapsto  \frac{1}{10} F(z) + \frac{9}{10} z$ on the same graph.  Do you think this fixed point sequence will converge?  Explain. 
 
 \part [5] Used fixed point iteration to find the fixed point of $z \mapsto  \frac{1}{10} F(z) + \frac{9}{10} z$.
 
 

 
\end{parts}
\end{questions}

\end{document}