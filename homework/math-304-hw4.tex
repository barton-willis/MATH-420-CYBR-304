\documentclass[12pt,fleqn]{exam}
\usepackage{pifont}
\usepackage{dingbat}
\usepackage{amsmath,amssymb}
\usepackage{fleqn}
\usepackage{epsfig}
%\usepackage{mathptm}
%\usepackage{euler}
\usepackage{bbding}
\usepackage{url}
\addpoints
\boxedpoints
\pointsinmargin
\pointname{pts}

\usepackage{xcolor}
\usepackage{framed}
\colorlet{shadecolor}{lightgray!15}
\newenvironment{myproof}
  {\begin{shaded}\begin{proof}}
  {\end{proof}\end{shaded}}
\shadedsolutions
\definecolor{SolutionColor}{rgb}{0.951,0.951,0.951}

\usepackage{amsthm}
\newtheorem{prop}{Proposition}
\usepackage[letterpaper, portrait, margin=0.75in]{geometry}
\usepackage[final]{microtype}
\usepackage[american]{babel}
%\usepackage[T1]{fontenc}
\usepackage{fourier}
\usepackage{isomath}
\usepackage{upgreek,amsmath}
\usepackage{amssymb}
%\usepackage[euler-digits,euler-hat-accent,T1]{eulervm}

\newcommand{\dotprod}{\, {\scriptzcriptztyle
    \stackrel{\bullet}{{}}}\,}

\newcommand{\reals}{\mathbf{R}}
\newcommand{\complex}{\mathbf{C}}
\newcommand{\dom}{\mbox{dom}}
\newcommand{\cover}{{\mathcal C}}
\newcommand{\rat}{\mathbf{Q}}

\newcommand{\curly}[1]{\mathcal #1}
\newcommand{\integers}{\mathbf{Z}}
\newcommand{\vi}{\, \mathbf{i}}
\newcommand{\vj}{\, \mathbf{j}}
\newcommand{\vk}{\, \mathbf{k}}
\newcommand{\bi}{\, \mathbf{i}}
\newcommand{\bj}{\, \mathbf{j}}
\newcommand{\bk}{\, \mathbf{k}}
\DeclareMathOperator{\Arg}{\mathrm{Arg}}
\DeclareMathOperator{\Ln}{\mathrm{Ln}}
\newcommand{\imag}{\, \mathrm{i}}
\newcommand{\range}{\mathrm{range}}
\newcommand{\true}{\mathrm{True}}
\newcommand{\saw}{\mathrm{saw}}
\newcommand{\fl}{\mathrm{Fl}}
\newcommand{\mach}{\varepsilon_m}
\newcommand{\HM}{\mathrm{HM}}
\newcommand{\Inf}{\mathrm{Inf}}
\usepackage{enumerate}
\usepackage{verbatim}
\newenvironment{alphalist}{
  \begin{enumerate}[(a)]
    \addtolength{\itemsep}{-0.5\itemsep}}
  {\end{enumerate}}
 
\usepackage{amsthm}
\newtheorem{Rubbish}{Theorem}
\usepackage{graphicx}

%\usepackage{tgschola} %to look retro
\newenvironment{mypar}[2]
  {\begin{list}{}%
    {\setlength\leftmargin{#1}
    \setlength\rightmargin{#2}}
    \item[]}
  {\end{list}}
  
\newcommand{\quiz}{3}
\newcommand{\term}{Spring}
\usepackage{units}
\usepackage{xspace}
\makeatletter
\DeclareRobustCommand{\maybefakesc}[1]{%
  \ifnum\pdfstrcmp{\f@series}{\bfdefault}=\z@
    {\fontsize{\dimexpr0.8\dimexpr\f@size pt\relax}{0}\selectfont\uppercase{#1}}%
  \else
    \textsc{#1}%
  \fi
}
\newcommand\AM{\,\maybefakesc{am}\xspace}
\newcommand\PM{\,\maybefakesc{pm}\xspace}

\begin{document}
\vspace{0.1in}
\noindent\makebox[3.0truein][l]{ \textbf{CYRR 304}}
{\bf Name:}  \\
\noindent \makebox[3.0truein][l]{\textbf{Homework \quiz, \term \/ \the\year}}
%{\bf Row:}\hrulefill\
\vspace{0.1in}

\begin{quote}
\emph{“Compound interest is the eighth wonder of the world. He who understands it, earns it; he who doesn't, pays it”.}  \hfill attributed to  {\sc Albert Einstein} 

\end{quote}
\noindent  Homework \quiz\/  has questions 1 through  \numquestions \/ with a total of of  \numpoints\/  points. 
Your recorded score will be scaled to twenty points. The point
value for each question or part of a question is in the box following each question or part of a question.
This work is due \textbf{Saturday 10 Feb  } at 11:59 \PM.  

For this assignment, convert your  Jupyter notebook (a IPYNB file)
 to HTML and submit the HTML file to Canvas. To convert you Jupyter notebook to HTML, do File --> Download as --> HTML. (For the File menu, look toward the 
 upper right corner.)
 
 \begin{questions}
 
 \question Starting on 1 January, Helen invests \$100 into stock mutual fund.  Every three months, she again invests \$100
 into the same fund.   The share price varies each day, so let's say that on the first of January the share price is $p_0$ dollars
 and the share price for the other three investments are $p_1, p_2$, and  $p_3$.  So on 31 December, the number of 
 shares Helen owns $S$, is given by
 \begin{equation}
     S = 100 \left (\frac{1}{p_0} + \frac{1}{p_1} + \frac{1}{p_2}  + \frac{1}{p_3} \right) ,
 \end{equation}
 and the value of these shares is  $\displaystyle p_4 S$. Helen would like to compare her investment to an investment that has a fixed rate of return of $r$. 
She calculates that had she similarly invested \$100 on the first day of each month to an investment with a fixed rate of 
return $r$, at the end of the year, her investment would be worth
\begin{equation}
    100 r \frac{(1+r)^{1/4}}{(1+r)^{1/4} - 1}.
\end{equation}
In official financial terms $r$ is the \emph{annual rate of return}, or APR.
So to find the APR of her mutual fund return, Helen wants to solve
\begin{equation}
 100 p_{4} \sum_{k=0}^{3} \frac{1}{p_k} = 100 r \frac{(1+r)^{1/4
\end{equation}
for $r$.  Of course, we can cancel the multiplicative factor of 100, so Patsy needs to solve the equation
\begin{equation}
 p_{4} \sum_{k=0}^3 \frac{1}{p_k} = r \frac{(1+r)^{1/4}}{(1+r)^{1/4} - 1}.
\end{equation}
for $r$.  The stuff on the left is known, so let's name it $q$.  So Patsy needs to solve
\begin{equation}
 q  = r \frac{(1+r)^{1/4}}{(1+r)^{1/4} - 1}.
\end{equation}
No amount of algebra tricks allow Patsy to solve this equation, so she decides to use a numerical
method.  Her favorite method is fixed point iteration, but her first step is to convert the equation to a fixed point form.
There are lots of ways to do this--we need to pick one and try it.  Here is one way
\begin{equation}
   r  = q \frac{(1+r)^{1/4} - 1}{(1+r)^{1/4} }
\end{equation}

 \begin{parts}
 
 \part [5] Use Gadfly to graph the function $r  \mapsto  r  \frac{(1+r)^{1/4}}{(1+r)^{1/4} - 1}$ on 
 the interval $r \in [-1,1]$.  With a return of 100\% (that is, $r = 1$), what is the account balance 
 on the last day of the year?
 
 \part [5] Suppose $q = 4.35$. (This means after investing a total of \$400 for the year, her account value at the end of the year is \$435, for a total gain of \$35 for the year.).  Use Gadfly to graph both $r \mapsto r$ and 
 $r \mapsto  4.35 \frac{(1+r)^{1/4} - 1}{(1+r)^{1/4} }$.  Based on this graph, do you expect that a fixed point sequence will converge?  Explain.
 
 \part [5] Write  a Julia function \texttt{APR} that takes the five share prices $p_0, p_1, p_2, p_3, p_4$ and returns
 the solution to the equation 
 \begin{equation}
 p_{4} \sum_{k=0}^3 \frac{1}{p_k} = r \frac{(1+r)^{1/4}}{(1+r)^{1/4} - 1}.
 \end{equation}
 To solve the equation, use fixed point iteration.   Check your answers against
 
\begin{verbatim}
julia> APY(1.0,1.0,1.0,1.0,1.0) 
         0.00


\end{verbatim}
 
\end{parts}
\end{questions}

\end{document}