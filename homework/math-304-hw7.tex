\documentclass[12pt,fleqn]{exam}
%\usepackage{pifont}
%\usepackage{dingbat}
\usepackage{amsmath,amssymb}
\usepackage{fleqn,nicefrac}
\usepackage{epsfig}
\usepackage{bbding}
\usepackage{url}
\addpoints
\boxedpoints
\pointsinmargin
\pointname{pts}

\usepackage{xcolor}
\usepackage{framed}
\colorlet{shadecolor}{lightgray!15}
\newenvironment{myproof}
  {\begin{shaded}\begin{proof}}
  {\end{proof}\end{shaded}}
\shadedsolutions
\definecolor{SolutionColor}{rgb}{0.951,0.951,0.951}

\usepackage{amsthm}
\newtheorem{prop}{Proposition}
\usepackage[letterpaper, portrait, margin=0.75in]{geometry}
\usepackage[final]{microtype}
\usepackage[american]{babel}
%\usepackage[T1]{fontenc}
\usepackage{fourier}
\usepackage{isomath}
\usepackage{upgreek,amsmath}
\usepackage{amssymb}
%\usepackage[euler-digits,euler-hat-accent,T1]{eulervm}

\newcommand{\dotprod}{\, {\scriptzcriptztyle
    \stackrel{\bullet}{{}}}\,}

\newcommand{\reals}{\mathbf{R}}
\newcommand{\complex}{\mathbf{C}}
\newcommand{\dom}{\mbox{dom}}
\newcommand{\cover}{{\mathcal C}}
\newcommand{\rat}{\mathbf{Q}}

\newcommand{\curly}[1]{\mathcal #1}
\newcommand{\integers}{\mathbf{Z}}
\newcommand{\vi}{\, \mathbf{i}}
\newcommand{\vj}{\, \mathbf{j}}
\newcommand{\vk}{\, \mathbf{k}}
\newcommand{\bi}{\, \mathbf{i}}
\newcommand{\bj}{\, \mathbf{j}}
\newcommand{\bk}{\, \mathbf{k}}
\DeclareMathOperator{\Arg}{\mathrm{Arg}}
\DeclareMathOperator{\Ln}{\mathrm{Ln}}
\newcommand{\imag}{\, \mathrm{i}}
\newcommand{\range}{\mathrm{range}}
\newcommand{\true}{\mathrm{True}}
\newcommand{\saw}{\mathrm{saw}}
\newcommand{\fl}{\mathrm{Fl}}
\newcommand{\mach}{\varepsilon_m}
\newcommand{\HM}{\mathrm{HM}}
\newcommand{\Inf}{\mathrm{Inf}}
\newcommand{\sigmoid}{\mathrm{sigmoid}}
\newcommand{\erf}{\mathrm{erf}}

\usepackage{enumerate}
\usepackage{verbatim}
\newenvironment{alphalist}{
  \begin{enumerate}[(a)]
    \addtolength{\itemsep}{-0.5\itemsep}}
  {\end{enumerate}}
 
\usepackage{amsthm}
\newtheorem{Rubbish}{Theorem}
\usepackage{graphicx}

%\usepackage{tgschola} %to look retro
\newenvironment{mypar}[2]
  {\begin{list}{}%
    {\setlength\leftmargin{#1}
    \setlength\rightmargin{#2}}
    \item[]}
  {\end{list}}
  
\newcommand{\quiz}{7}
\newcommand{\term}{Spring}
\usepackage{units}
\usepackage{xspace}
\makeatletter
\DeclareRobustCommand{\maybefakesc}[1]{%
  \ifnum\pdfstrcmp{\f@series}{\bfdefault}=\z@
    {\fontsize{\dimexpr0.8\dimexpr\f@size pt\relax}{0}\selectfont\uppercase{#1}}%
  \else
    \textsc{#1}%
  \fi
}
\newcommand\AM{\,\maybefakesc{am}\xspace}
\newcommand\PM{\,\maybefakesc{pm}\xspace}


\usepackage{fancyvrb,newverbs,xcolor}
\usepackage{lipsum}% just for this example

\definecolor{cverbbg}{gray}{0.93}

\newenvironment{cverbatim}
 {\SaveVerbatim{cverb}}
 {\endSaveVerbatim
  \flushleft\fboxrule=0pt\fboxsep=.5em
  \colorbox{cverbbg}{\BUseVerbatim{cverb}}%
  \endflushleft
}
\newenvironment{lcverbatim}
 {\SaveVerbatim{cverb}}
 {\endSaveVerbatim
  \flushleft\fboxrule=0pt\fboxsep=.5em
  \colorbox{cverbbg}{%
    \makebox[\dimexpr\linewidth-2\fboxsep][l]{\BUseVerbatim{cverb}}%
  }
  \endflushleft
}

\newcommand{\ctexttt}[1]{\colorbox{cverbbg}{\texttt{#1}}}
\newverbcommand{\cverb}
  {\setbox\verbbox\hbox\bgroup}
  {\egroup\colorbox{cverbbg}{\box\verbbox}}
\begin{document}
\vspace{0.1in}
\noindent\makebox[3.0truein][l]{ \textbf{CYRR 304}}
{\bf Name:}  \\
\noindent \makebox[3.0truein][l]{\textbf{Homework \quiz, \term \/ \the\year}}
%{\bf Row:}\hrulefill\
\vspace{0.1in}

\begin{quote}
\noindent \emph{“When you’re good to others, you’re best to yourself.”} \hfill {\sc Benjamin Franklin}
\end{quote}

\noindent  Homework \quiz\/  has questions 1 through  \numquestions \/ with a total of of  \numpoints\/  points. 
Your recorded score will be scaled to twenty points. The point
value for each question or part of a question is in the box following each question or part of a question.
This work is due \textbf{Saturday 23 March  } at 11:59 \PM.  

For this assignment, convert your Jupyter notebook (a IPYNB file)  to HTML and submit the HTML file to Canvas.

The degree four Bernstein polynomials $P_0$ through $P_4$ are defined by
\begin{equation}
P_0(x) = (1-x)^4, \quad P_1(x)  = 4 x (1-x)^3, \quad P_2(x) = 6 x^2 (1-x)^2, \quad P_3(x) = 4 x^3 (1-x),
\quad P_4(x) = x^4.
\end{equation}
The degree of each Bernstein polynomial is four, and it can be proven that every polynomial of degree four or less is a linear
combination of these five polynomials.  On the interval $[0,1]$, the Bernstein polynomials are non negative. This 
property makes them well suited to approximating functions that are nonnegative on $[0,1]$.  
\begin{questions}

\question [5] Use Gadfly to graph $P_0$ through $P_1$ on the interval $[0,1]$.

\question [5] Find the real numbers $c_0, c_1, c_2,c_3, c_4,$ and $c_5$ such that the function 
\begin{equation}
  F(x) = \sum_{k=0}^5 c_k P_k(x)
\end{equation}
minimizes
\begin{equation}
   \sum_{k=0}^5 \left(c_k P \left (\frac{k}{10} \right) - \sin\left(\frac{\uppi k}{10}\right) \right)^2.
 \end{equation}
 Be sure to find the condition number of the coefficient matrix of the normal equations.

\question [5] Use Gadfly to graph both $y = \sin(\uppi x)$ and $ y = \sum_{k=0}^5 c_k P_k(x)$ on the interval $[0,1]$, where the numbers
$c_0$ through $c_5$ are the numbers that you found in Question 2.

\question [5] Use Gadfly to graph  $y = \left| \sin(\uppi x) -  \sum_{k=0}^5 c_k P_k(x) \right| $, where the numbers
$c_0$ through $c_5$ are the numbers that you found in Question 2.  Visually find $\underset{x \in [0,1]}{\max} 
\left(\left| \sin(\uppi x) -  \sum_{k=0}^5 c_k P_k(x) \right| \right).$
\end{questions}

\end{document}
