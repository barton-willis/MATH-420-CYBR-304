\documentclass[12pt,fleqn]{exam}
\usepackage{pifont}
\usepackage{dingbat}
\usepackage{amsmath,amssymb}
\usepackage{fleqn}
\usepackage{epsfig}
%\usepackage{mathptm}
%\usepackage{euler}
\usepackage{bbding}
\usepackage{url}
\addpoints
\boxedpoints
\pointsinmargin
\pointname{pts}

\usepackage{xcolor}
\usepackage{framed}
\colorlet{shadecolor}{lightgray!15}
\newenvironment{myproof}
  {\begin{shaded}\begin{proof}}
  {\end{proof}\end{shaded}}
\shadedsolutions
\definecolor{SolutionColor}{rgb}{0.951,0.951,0.951}

\usepackage{amsthm}
\newtheorem{prop}{Proposition}
\usepackage[letterpaper, portrait, margin=0.75in]{geometry}
\usepackage[final]{microtype}
\usepackage[american]{babel}
%\usepackage[T1]{fontenc}
\usepackage{fourier}
\usepackage{isomath}
\usepackage{upgreek,amsmath}
\usepackage{amssymb}
%\usepackage[euler-digits,euler-hat-accent,T1]{eulervm}

\newcommand{\dotprod}{\, {\scriptzcriptztyle
    \stackrel{\bullet}{{}}}\,}

\newcommand{\reals}{\mathbf{R}}
\newcommand{\complex}{\mathbf{C}}
\newcommand{\dom}{\mbox{dom}}
\newcommand{\cover}{{\mathcal C}}
\newcommand{\rat}{\mathbf{Q}}

\newcommand{\curly}[1]{\mathcal #1}
\newcommand{\integers}{\mathbf{Z}}
\newcommand{\vi}{\, \mathbf{i}}
\newcommand{\vj}{\, \mathbf{j}}
\newcommand{\vk}{\, \mathbf{k}}
\newcommand{\bi}{\, \mathbf{i}}
\newcommand{\bj}{\, \mathbf{j}}
\newcommand{\bk}{\, \mathbf{k}}
\DeclareMathOperator{\Arg}{\mathrm{Arg}}
\DeclareMathOperator{\Ln}{\mathrm{Ln}}
\newcommand{\imag}{\, \mathrm{i}}
\newcommand{\range}{\mathrm{range}}
\newcommand{\true}{\mathrm{True}}
\newcommand{\saw}{\mathrm{saw}}
\newcommand{\fl}{\mathrm{Fl}}
\newcommand{\mach}{\varepsilon_m}
\newcommand{\HM}{\mathrm{HM}}
\newcommand{\Inf}{\mathrm{Inf}}
\usepackage{enumerate}
\usepackage{verbatim}
\newenvironment{alphalist}{
  \begin{enumerate}[(a)]
    \addtolength{\itemsep}{-0.5\itemsep}}
  {\end{enumerate}}
 
\usepackage{amsthm}
\newtheorem{Rubbish}{Theorem}
\usepackage{graphicx}

%\usepackage{tgschola} %to look retro
\newenvironment{mypar}[2]
  {\begin{list}{}%
    {\setlength\leftmargin{#1}
    \setlength\rightmargin{#2}}
    \item[]}
  {\end{list}}
  
\newcommand{\quiz}{3}
\newcommand{\term}{Spring}
\usepackage{units}
\usepackage{xspace}
\makeatletter
\DeclareRobustCommand{\maybefakesc}[1]{%
  \ifnum\pdfstrcmp{\f@series}{\bfdefault}=\z@
    {\fontsize{\dimexpr0.8\dimexpr\f@size pt\relax}{0}\selectfont\uppercase{#1}}%
  \else
    \textsc{#1}%
  \fi
}
\newcommand\AM{\,\maybefakesc{am}\xspace}
\newcommand\PM{\,\maybefakesc{pm}\xspace}

\begin{document}
\large
\vspace{0.1in}
\noindent\makebox[3.0truein][l]{ \textbf{CYRR 304}}
{\bf Name:}  \\
\noindent \makebox[3.0truein][l]{\textbf{Homework \quiz, \term \/ \the\year}}
%{\bf Row:}\hrulefill\
\vspace{0.1in}

\begin{quote}
“\emph{Fast is fine, but accuracy is everything.}”  \hfill {\sc Wyatt Erp} 

\end{quote}
\noindent  Homework \quiz\/  has questions 1 through  \numquestions \/ with a total of of  \numpoints\/  points. 
Your recorded score will be scaled to twenty points. The point
value for each question or part of a question is in the box following each question or part of a question.
This work is due \textbf{Saturday 10 Feb  } at 11:59 \PM.  

For this assignment, convert your  Jupyter notebook (a IPYNB file)
 to HTML and submit the HTML file to Canvas. To convert you Jupyter notebook to HTML, do File --> Download as --> HTML. (For the File menu, look toward the 
 upper right corner.)
 
 \begin{questions}
 
 \question Our problem this week comes from celestial mechanics. For a comet in parabolic orbit, the final step in finding the position of the comet as a function of time is to find the real solution to the equation $y^3 + 3y - x = 0$.  The (and indeed there is only one) real solution of this
 equation has a nice form; it is
 \begin{equation}
 y={{\left( \frac{\sqrt{{{x}^{2}}+4}}{2}+\frac{x}{2}\right) }^{\frac{1}{3}}}-\frac{1}{{{\left( \frac{\sqrt{{{x}^{2}}+4}}{2}+\frac{x}{2}\right) }^{\frac{1}{3}}}}
 \end{equation}
 
 \begin{parts}
 
 \part [5] Write a Julia function \texttt{orbit} that takes a number $x$ as its input and returns
 \begin{equation*}
 {{\left( \frac{\sqrt{{{x}^{2}}+4}}{2}+\frac{x}{2}\right) }^{\frac{1}{3}}}-\frac{1}{{{\left( \frac{\sqrt{{{x}^{2}}+4}}{2}+\frac{x}{2}\right) }^{\frac{1}{3}}}}
 \end{equation*}
 as its output.   Do two quick checks on your code and verify that \texttt{orbit(42.0) = 3.1890525141322277} and \texttt{orbit(-107.0) = -4.536964755824514}. If the last few digits of your value differ, don't panic. There can be 
 some differences due to microprocessors and the operating system.
 
 
 \part [5] Evaluate \texttt{orbit(-1.0e8)}.
 
 \part [5] Use the bisection method to solve the equation $x^3 + 3 x + 10^8 = 0$. This should give 
 the same value as  \texttt{orbit(-1.0e8)}.  For the bisection method in Julia, you may use the code we
 used in class, or write your own. For your initial interval,   try $[-500,-100]$. Does this agree 
 with the value from part `b`?
 
 
 \part [5] Draw a graph of the function \texttt{orbit}.  Draw the graph for inputs from $-10^8$ to $10^8$.
 From our knowledge of calculus, the graph should be smooth and continuous. Does it appear to be smooth and
 continuous?  Explain.
 
 \end{parts}
 
 \end{questions}
 What's the story? For $x \ll -1$, numerical evaluation of $\frac{\sqrt{{{x}^{2}}+4}}{2}+\frac{x}{2}$ is a trouble
 maker.  Why so? Let $x = -10^8$. We need to evaluate
 \begin{equation}
 \frac{\sqrt{10^{16}+4}}{2}-\frac{10^8}{2}.
\end{equation}
But this is the difference of two numbers that are very close together, so the condition number of this sum is huge.
This means that the relative difference of the true value of this sum and the computed value can be large.  
And what are the lessons? First, adding two floating point numbers that have opposite signs and nearly equal magnitudes
can give large relative errors. Second, ill-conditioned sums  aren't always easy to spot. And third, as intriguing and beautiful as the exact solution of the cubic maybe,
it is poorly suited for numerical evaluation and a \emph{numerical method is more accurate}. 




 

\end{document}