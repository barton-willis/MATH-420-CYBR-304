\documentclass[12pt,fleqn]{exam}
%\usepackage{pifont}
%\usepackage{dingbat}
\usepackage{amsmath,amssymb}
\usepackage{fleqn}
\usepackage{epsfig}
\usepackage{bbding}
\usepackage{url}
\addpoints
\boxedpoints
\pointsinmargin
\pointname{pts}

\usepackage{xcolor}
\usepackage{framed}
\colorlet{shadecolor}{lightgray!15}
\newenvironment{myproof}
  {\begin{shaded}\begin{proof}}
  {\end{proof}\end{shaded}}
\shadedsolutions
\definecolor{SolutionColor}{rgb}{0.951,0.951,0.951}

\usepackage{amsthm}
\newtheorem{prop}{Proposition}
\usepackage[letterpaper, portrait, margin=0.75in]{geometry}
\usepackage[final]{microtype}
\usepackage[american]{babel}
%\usepackage[T1]{fontenc}
\usepackage{fourier}
\usepackage{isomath}
\usepackage{upgreek,amsmath}
\usepackage{amssymb}
%\usepackage[euler-digits,euler-hat-accent,T1]{eulervm}

\newcommand{\dotprod}{\, {\scriptzcriptztyle
    \stackrel{\bullet}{{}}}\,}

\newcommand{\reals}{\mathbf{R}}
\newcommand{\complex}{\mathbf{C}}
\newcommand{\dom}{\mbox{dom}}
\newcommand{\cover}{{\mathcal C}}
\newcommand{\rat}{\mathbf{Q}}

\newcommand{\curly}[1]{\mathcal #1}
\newcommand{\integers}{\mathbf{Z}}
\newcommand{\vi}{\, \mathbf{i}}
\newcommand{\vj}{\, \mathbf{j}}
\newcommand{\vk}{\, \mathbf{k}}
\newcommand{\bi}{\, \mathbf{i}}
\newcommand{\bj}{\, \mathbf{j}}
\newcommand{\bk}{\, \mathbf{k}}
\DeclareMathOperator{\Arg}{\mathrm{Arg}}
\DeclareMathOperator{\Ln}{\mathrm{Ln}}
\newcommand{\imag}{\, \mathrm{i}}
\newcommand{\range}{\mathrm{range}}
\newcommand{\true}{\mathrm{True}}
\newcommand{\saw}{\mathrm{saw}}
\newcommand{\fl}{\mathrm{Fl}}
\newcommand{\mach}{\varepsilon_m}
\newcommand{\HM}{\mathrm{HM}}
\newcommand{\Inf}{\mathrm{Inf}}
\usepackage{enumerate}
\usepackage{verbatim}
\newenvironment{alphalist}{
  \begin{enumerate}[(a)]
    \addtolength{\itemsep}{-0.5\itemsep}}
  {\end{enumerate}}
 
\usepackage{amsthm}
\newtheorem{Rubbish}{Theorem}
\usepackage{graphicx}

%\usepackage{tgschola} %to look retro
\newenvironment{mypar}[2]
  {\begin{list}{}%
    {\setlength\leftmargin{#1}
    \setlength\rightmargin{#2}}
    \item[]}
  {\end{list}}
  
\newcommand{\quiz}{5}
\newcommand{\term}{Spring}
\usepackage{units}
\usepackage{xspace}
\makeatletter
\DeclareRobustCommand{\maybefakesc}[1]{%
  \ifnum\pdfstrcmp{\f@series}{\bfdefault}=\z@
    {\fontsize{\dimexpr0.8\dimexpr\f@size pt\relax}{0}\selectfont\uppercase{#1}}%
  \else
    \textsc{#1}%
  \fi
}
\newcommand\AM{\,\maybefakesc{am}\xspace}
\newcommand\PM{\,\maybefakesc{pm}\xspace}


\usepackage{fancyvrb,newverbs,xcolor}
\usepackage{lipsum}% just for this example

\definecolor{cverbbg}{gray}{0.93}

\newenvironment{cverbatim}
 {\SaveVerbatim{cverb}}
 {\endSaveVerbatim
  \flushleft\fboxrule=0pt\fboxsep=.5em
  \colorbox{cverbbg}{\BUseVerbatim{cverb}}%
  \endflushleft
}
\newenvironment{lcverbatim}
 {\SaveVerbatim{cverb}}
 {\endSaveVerbatim
  \flushleft\fboxrule=0pt\fboxsep=.5em
  \colorbox{cverbbg}{%
    \makebox[\dimexpr\linewidth-2\fboxsep][l]{\BUseVerbatim{cverb}}%
  }
  \endflushleft
}

\newcommand{\ctexttt}[1]{\colorbox{cverbbg}{\texttt{#1}}}
\newverbcommand{\cverb}
  {\setbox\verbbox\hbox\bgroup}
  {\egroup\colorbox{cverbbg}{\box\verbbox}}
\begin{document}
\vspace{0.1in}
\noindent\makebox[3.0truein][l]{ \textbf{CYRR 304}}
{\bf Name:}  \\
\noindent \makebox[3.0truein][l]{\textbf{Homework \quiz, \term \/ \the\year}}
%{\bf Row:}\hrulefill\
\vspace{0.1in}

\begin{quote}
\emph{“Is it hard?' Not if you have the right attitudes. It's having the right attitudes that's hard.”} \\ $\phantom{xxx} $ \hfill {\sc Robert Pirsig}
\end{quote}

\noindent  Homework \quiz\/  has questions 1 through  \numquestions \/ with a total of of  \numpoints\/  points. 
Your recorded score will be scaled to twenty points. The point
value for each question or part of a question is in the box following each question or part of a question.
This work is due \textbf{Saturday 2 March  } at 11:59 \PM.  

For this assignment, convert your Jupyter notebook (a IPYNB file)  to HTML and submit the HTML file to Canvas.

\begin{questions}

\question [10] Write a Julia function \texttt{PR} (page rank) that does fixed point iteration on a stochastic matrix.   This is a
simplified version of the Google Page Rank algorithm. Specifically,  for a stochastic matrix
$M$, the function should find the terms of the recursively defined sequence $p$ defined as
\begin{equation*}
    p_{k} = \begin{cases} p_0 & k = 0 \\    M p_{k-1}  & k  \in \integers_{\geq 1}  \end{cases}.
\end{equation*}
For $p_0$, use a column vector whose elements are all given by $1/n$, where $n$ is the number of rows of the matrix $M$.  The Julia function \texttt{size} returns the size of a matrix as a two-tuple; for example

\begin{cverbatim}
julia> size([1/2 1/3 0; 0 1/3 1/2; 1/2 1/3 1/2])
       (3,3)
\end{cverbatim}
To construct $p_0$, use a Julia array comprehension;  for example
\begin{cverbatim}
julia> mat = [1/2 1/3 0; 0 1/3 1/2; 1/2 1/3 1/2];

julia> (m,n) = size(mat);

julia> p0 = [[1/m] for k=1:m]

3-element Vector{Vector{Float64}}:
 [0.3333333333333333]
 [0.3333333333333333]
 [0.3333333333333333]

\end{cverbatim}

The function \texttt{PR} should stop finding terms of the fixed point sequence when consecutive terms $p_k$ and $p_{k+1}$ satisfy 
$\| p_k - p_{k+1} \|_{\infty}  < \text{tol}$, where the positive number $\text{tol}$ is a user-supplied tolerance.  To find the infinity norm, use the function \texttt{norm}. To use this function, you will need
to use the Julia package manager to install the package \texttt{LinearAlgebra}.  After you have installed and loaded this package, 
you can find the infinity norm using 
\begin{cverbatim}
julia> norm([[1],[-4], [-2024]],Inf)
     2024
\end{cverbatim}

When the stopping criteria is never satisfied, your code should give an error ``Fixed point sequence diverges or converges too slowly.''

Let's give our function function keyword arguments to both the stopping tolerance \texttt{tol} and for the maximum number of
iterations \texttt{maxiter}. Specifically, define your function as

\begin{cverbatim}
function PR(mat::Matrix; tol = sqrt(eps(eltype(mat))), maxiter::Int64 = 50)
\end{cverbatim}

\question[10] Test your function \texttt{PR} on the four inputs
\begin{cverbatim}
PR([1/2 1/3 0; 0 1/3 1/2; 1/2 1/3 1/2])
3-element Vector{Vector{Float64}}:
 [0.22222222753502474]
 [0.3333333285777581]
 [0.4444444438872168]
 
 PR([0.0 1.0; 1.0 0.0])
 2-element Vector{Vector{Float64}}:
 [0.5]
 [0.5]
 
 PR( [1 0.1 0 0 ; 0 0 0 1; 0 0 1 0; 0 0.9 0 0])
 Fixed point sequence diverges or converges too slowly.
 
 PR( [1 0.1 0 0 ; 0 0 0 1; 0 0 1 0; 0 0.9 0 0], maxiter=1000)
 4-element Vector{Vector{Float64}}:
 [0.7499997441900677]
 [1.3463680648771707e-7]
 [0.25]
 [1.2117312583894537e-7]
 
\end{cverbatim}

Your values should be very close to the given values; if not, check your work to part `a,' and try again.

\end{questions}

\end{document}
