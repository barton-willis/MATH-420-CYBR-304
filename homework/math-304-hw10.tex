\documentclass[12pt,fleqn]{exam}
\usepackage{amsmath,fleqn, url}
\usepackage{enumerate,color}
%\pagestyle{empty}

\usepackage{graphicx}
\usepackage[letterpaper,margin=1.0in]{geometry}
\usepackage[english]{babel}
\usepackage[final,babel]{microtype}
\usepackage{amsmath, amssymb, amsthm, upgreek, xfrac, isomath}
\frenchspacing
\usepackage[T1]{fontenc}
%\usepackage{lmodern}
\usepackage[upright]{fourier} %options: sloped, upright, widespace
%\usepackage[euler-digits,euler-hat-accent,T1]{eulervm}

\newcommand{\reals}{\mathbf{R}}
\newcommand{\integers}{\mathbf{Z}}
\newcommand{\eu}{\mathrm{e}}
\newcommand{\range}{\mathrm{range}}
\usepackage{algorithm}
\usepackage[noend]{algorithmic}
\usepackage{isomath}
\usepackage{graphicx}
\renewcommand{\algorithmicrequire}{\textbf{input}}
\renewcommand{\algorithmicensure}{\textbf{return}}

\usepackage[T1]{fontenc}


\usepackage[usenames,dvipsnames]{xcolor}


\newcommand{\algo}[1]{\vspace{0.1in}
      \begin{flushleft}
         {\bf Algorithm} {\em #1}
      \end{flushleft}
      \vspace{0.1in}}

\newenvironment{myalgorithm}[1]
  {\setcounter{myalgo}{\value{myalgo}+1} 
   \textbf{Algorithm \themysec.\hspace{-0.15in}\myalgo} {\em #1} \\ \vspace{-0.25in}}
  

\DeclareMathOperator{\fl}{fl}
\newenvironment{alphalist}{
  \begin{enumerate}[(a)]
    \addtolength{\itemsep}{-1.0\itemsep}}
  {\end{enumerate}}

\newcommand{\quiz}{10}
\newcommand{\term}{Spring}
\usepackage{units}
\usepackage{xspace}
\makeatletter
\DeclareRobustCommand{\maybefakesc}[1]{%
  \ifnum\pdfstrcmp{\f@series}{\bfdefault}=\z@
    {\fontsize{\dimexpr0.8\dimexpr\f@size pt\relax}{0}\selectfont\uppercase{#1}}%
  \else
    \textsc{#1}%
  \fi
}
\newcommand\AM{\,\maybefakesc{am}\xspace}
\newcommand\PM{\,\maybefakesc{pm}\xspace}

\addpoints
\boxedpoints
\pointsinmargin
\pointname{pts}
\begin{document}

\vspace{0.1in}
\noindent\makebox[3.0truein][l]{ \textbf{CYRR 304}}
{\bf Name:}  \\
\noindent \makebox[3.0truein][l]{\textbf{Homework \quiz, \term \/ \the\year}}
%{\bf Row:}\hrulefill\
\vspace{0.1in}

\noindent  Homework 10  has questions 1 through  \numquestions \/ with a total of of  \numpoints\/  points. The point
value for each question or part of a question is in the box following each question or part of a question.
This work is due \textbf{Saturday 20 April  } at 11:59 \PM.  

In class (and in our textbook), we applied the Gram-Schmidt process to the functions $x \mapsto 1$, $x \mapsto x, x \mapsto x^2, \dots$ using the interval $[-1,1]$ and an inner product
defined as 
\begin{equation*}
\langle f,g \rangle = \int_{-1}^1 f(x) g(x) \, \mathrm{d} x.
\end{equation*}
For this homework, we'll do the same, except that we will change the inner product to 
\begin{equation*}
 \langle f,g \rangle = \int_{-1}^1 \sqrt{1-x^2} f(x) g(x) \, \mathrm{d} x.
\end{equation*}
This inner product has all the required properties and the weight function satisfies the requirements on page 205 of our textbook.

Doing the Gram-Schimidt process by hand for these functions will develop strength and  character, but it's tedious and challenging to get correct. The result of doing so is
\begin{align*}
C_0(x) &= \frac{\sqrt{2}}{\sqrt{\uppi}}, \\
C_1(x) &= \frac{\sqrt{8}}{\sqrt{\uppi}} x,\\
C_2(x) &= \frac{\sqrt{2}}{\sqrt{\uppi}} (2x-1)(2x+1), \\
C_3(x) &= \frac{\sqrt{32}}{\sqrt{\uppi}} x(2x^2-1), \\
C_4(x) &= \frac{\sqrt{2}}{\sqrt{\uppi}} (4 x^2 - 2 x - 1)(4x^4+2x+1). 
\end{align*}
These functions have been normalized so that $ \langle C_k, C_\ell  \rangle  = \begin{cases} 0 & k \neq \ell \\ 1 & k=\ell \end{cases}$.  These functions are a special case
of the well-known \emph{ultraspherical} polynomials that are known as the \emph{Chebyshev polynomials of the second kind}. The zeros of $C_4$ are 
\begin{equation*}
x_0 =  \operatorname{-}\left( \frac{\sqrt{5}\operatorname{+}1}{4}\right) ,
x_1 = \operatorname{-}\left( \frac{\sqrt{5}\operatorname{-}1}{4}\right),
x_2 = \frac{\sqrt{5}\operatorname{-}1}{4}\operatorname{,}
x_3 = \frac{\sqrt{5}\operatorname{+}1}{4}
 \end{equation*}

\begin{questions}

\question[10] These polynomials can be defined recursively by 
\begin{equation*}
 C_{n+1}(x) = 2  x \, C_n(x) - C_{n-1} (x),
\end{equation*}
and $C_0(x) = \frac{\sqrt{2}}{\sqrt{\uppi}},  C_1(x) = \frac{\sqrt{8}}{\sqrt{\uppi}} x$.  Write a Julia function \texttt{ ultraspherical(n::Integer, x::Number)}  that evaluates these polynomials. 



\question[10]  Check that $C_{10}(0.23) \approx 0.6818339026993401$  and  $C_{15}(-0.19) \approx 0.06737434860812452$

\question[10] Use Gadfly to graph $C_0, C_1, C_2, C_3$, and $C_4$ on the interval $[-1,1]$.


\question [10] As we did for the Legendre polynomials, let's build a quadrature rule whose knots are the zeros of $C_4$. The corresponding weights are

\begin{equation*}
 \ensuremath{\mathrm{w0}}\operatorname{=}\operatorname{-}\left( \frac{\left( \sqrt{5}\operatorname{-}5\right)  \ensuremath{\pi} }{40}\right) \operatorname{,}\ensuremath{\mathrm{w1}}\operatorname{=}\frac{\left( \sqrt{5}\operatorname{+}5\right)  \ensuremath{\pi} }{40}\operatorname{,}\ensuremath{\mathrm{w2}}\operatorname{=}\frac{\left( \sqrt{5}\operatorname{+}5\right)  \ensuremath{\pi} }{40}\operatorname{,}\ensuremath{\mathrm{w3}}\operatorname{=}\operatorname{-}\left( \frac{\left( \sqrt{5}\operatorname{-}5\right)  \ensuremath{\pi} }{40}\right) 
\end{equation*}
The resulting quadrature rule is
\begin{equation}
 \int_{-1}^1 \sqrt{1-x^2} F(x) \, \mathrm{d} x \approx \sum_{k=0}^3 w_k F(x_k).
\end{equation}
Write a Julia function that implements this quadrature rule.

\question [10] Show that this quadrature rule is pretty close to exact when $F$ is a polynomial of degree seven or less.  To do this, test the rule on each of the following:
\begin{align*}
&\int_{-1}^{1}{\sqrt{1-x^2}\;\mathrm{d}x}={{\pi}\over{2}} , \\
&\int_{-1 }^{1}{x\,\sqrt{1-x^2}\;\mathrm{d}x}=0 , \\
&\int_{-1}^{1}{x^2\,\sqrt{1-x^2}\;\mathrm{d}x}=  {{\pi}\over{8}} , \\
 &\int_{-1}^{1}{x^3\,\sqrt{1-x^2}\;\mathrm{d}x}=0 , \\
 &\int_{-1 }^{1}{x^4\,\sqrt{1-x^2}\;\mathrm{d}x}={{\pi}\over{16}} , \\
  &\int_{-1}^{1}{x^5\, \sqrt{1-x^2}\;\mathrm{d}x}=0 , \\
   &\int_{-1}^{1}{x^6\,\sqrt{1-x^2}\;\mathrm{d}x}={{5\,\pi  }\over{128}} , \\
   &\int_{-1}^{1}{x^7\,\sqrt{1-x^2}\;\mathrm{d}x}=0. 
 \end{align*}

\end{questions}
\end{document}