\documentclass[portrait,fleqn,12pt]{beamer}
\usetheme[progressbar=frametitle]{metropolis}
\usepackage{amsmath}
\usepackage{fleqn,amssymb,shadow}
\usepackage{float}
\usepackage{epsfig}
\usepackage{enumerate}
\usepackage{cancel}
\usepackage{fontawesome5}
%\usepackage{fourier}
\usepackage{upgreek}
%\raggedright

% Custom commands
\newcommand{\fma}{{\rm fma}}
\newcommand{\reals}{\mathbf{R}}
\newcommand{\epsmach}{\varepsilon_m}
\newcommand{\mymax}{\mbox{max} \,\,}
\newcommand{\mydeg}{\mbox{deg}}
\newcommand{\dom}{\mbox{dom}}
\newcommand{\fl}{\mbox{fl}}
\newcommand{\rel}{\mbox{rel}}
\newcommand{\cond}{\mbox{cond}}
\newcommand{\bern}{\mathcal{B}}
\newcommand{\order}{\mathcal{O}}
\newcommand{\atan}{\rm{arctan}}
\newcommand{\bfa}{\mathbf{a}}
\newcommand{\bfb}{\mathbf{b}}
\newcommand{\bfzero}{\mathbf{0}}
\newcommand{\bfx}{\mathbf{x}}
\newcommand{\complex}{\mathbf{C}}
\newcommand{\integers}{\mathbf{Z}}
\newcommand{\D}{\mathrm{D}}

\newcommand{\nth}{n\textsuperscript{th}\,}
\newcommand{\kth}{k\textsuperscript{th}\,}
% Custom environments
\newenvironment{alphalist}
   {\begin{enumerate}[(a)]
       \addtolength{\itemsep}{0.0\itemsep}}
     {\end{enumerate}}

\newenvironment{handlist}
   {\begin{enumerate}[\faHandPointRight]
       \addtolength{\itemsep}{0.0\itemsep}}
     {\end{enumerate}}

\newenvironment{define}[1]{
  \textbf{Definition} #1}{}

\newenvironment{idefine}[2]{
  \index{#1}
  \textbf{Definition} #2}{\(\blacktriangleleft\)}

\newenvironment{myexample}[1]{
  \textbf{Example} #1}


\newenvironment{mynote}[1]{
  \textbf{Note} #1}

\begin{document}

\begin{frame}
\begin{flushleft} 
\textbf{Fun with generating functions} \\
MATH 420 \& CYBR 304 \\
Spring 2024 
\end{flushleft}


\emph{“We have no reason to believe a priori this series will exist, but we will be optimistic.”} \hfill{\sc Donald Knuth}

\emph{“If you have a good recipe and can measure accurately, you can bake it.”} \hfill{\sc Frances Phillips}
\end{frame}


\begin{frame}[fragile]{Our problem}
We would like to find an explicit formula for the sequence $c$ that is defined recursively by
\begin{equation}
  c_n= \begin{cases} 0 & n =0 \text{ or } n =1 \\
                                 c_{n-1} + c_{n-2}+ 8 & n \geq 2
           \end{cases}.
\end{equation}
\begin{handlist}
\item The first eleven terms of the sequence $c$ are
\begin{equation}
      0,0,8,16,32,56,96,160,264,432,704.
\end{equation}
\item To complete this task, we need \emph{four} new tools.
\end{handlist}
\end{frame}
\begin{frame}{Tool 1:  Binomial Coefficients.}

For positive integers $n$ and $k$ with $n \geq k$, we define
\begin{equation}
      \binom{n}{k} = \frac{n!}{k! (n-k)!}.
\end{equation}
\begin{handlist}
 \item Spoken $\binom{n}{k}$ is ``$n$ choose $k$.''
 \item For a set $A$ with cardinality n,  the number of subsets of $A$ that have cardinality $k$ is $\binom{n}{k}$.
 \item The cardinality two subsets of $\{1,2,3\}$ are $\{1,2\},\{1,3\},$ and $\{2,3\}$.  So  $\binom{3}{2} = 3$.
\end{handlist}
\end{frame}

\begin{frame}{Tool 2:  Product rule for \nth derivatives}

For smooth functions $f$ and $g$ and a nonnegative integer $n$, the \nth derivative of the product of $f$ times $g$ is
\begin{equation}
       \D^n (f g) = \sum_{k=0}^n  \binom{n}{k}  f^{(k)} g^{(n-k)}.
\end{equation}
\begin{handlist}
\item $ f^{(k)}$ is the \kth derivative of $f$, not the \kth power.
\item This is a generalization of the product rule for derivatives.
\item $(f g)^{\prime \prime} = \binom{2}{0} f g^{\prime \prime} + \binom{2}{1} f^\prime  g^{\prime} +
\binom{2}{2} f^{\prime \prime} g  = f g^{\prime \prime} + 2  f^\prime  g^{\prime} + f^{\prime \prime} g$.
\end{handlist}
\end{frame}


\begin{frame}{Tool 3:  \nth derivative of a power series}

Let $c$ be a sequence and let $n$ be a nonnegative integer.  Then
\begin{equation}
      \left.  \D^n_x \left( \sum_{k=0}^\infty c_k x^k  \right)  \right \vert_{x=0} = n!  c_n.
\end{equation}
\begin{handlist}
 \item The operator order is sum first, derivative second, and  evaluation at zero third.
   \item For this to be true, it has to be the case that  the radius of convergence of the series is nonzero.
\end{handlist}
\end{frame}

\begin{frame}{Tool 4:   \nth derivative of a rational function}

Let $a \in \reals_{\neq 0} $ and $n \in \integers_{\geq 0}$. Then
\begin{equation*}
      \left.  \D^n_x \left(\frac{1}{x-a}  \right) \right \vert_{x=0} = -\frac{n!}{a^{n+1}}.
\end{equation*}
And let $a,b \in \reals_{\neq 0} $ with $a \neq b$  and let $n \in \integers_{\geq 0}$. Then
\begin{equation*}
      \left.  \D^n_x \left(\frac{1}{(x-a)(x-b)}  \right) \right \vert_{x=0} = -\frac{n!}{a-b} \left(\frac{1}{a^{n+1}} - \frac{1}{b^{n+1}} \right)
\end{equation*}
\end{frame}

\begin{frame}{Step 0}

Multiply the recursion $c_n = c_{n-1} + c_{n-2}+ 8$ by $z^n$ and sum from $n=2$ to $n=\infty$.  This gives
\begin{equation}
  \sum_{n=2}^\infty c_{n} z^n  =  \sum_{n=2}^\infty c_{n-1} z^n +  \sum_{n=2}^\infty c_{n-2} z^n +  \sum_{n=2}^\infty 8 z^n
\end{equation}
\begin{handlist}
\item The lowest sum index is two because the recursion  $c_n = c_{n-1} + c_{n-2}+ 8$ is only valid for $n \geq 2$.
\end{handlist}
\end{frame}
\begin{frame}{Step 1}

If needed, shift each sum index to make each summand involve $c_n$, not $c_{n-1}$ or $c_{n-2}$.
\begin{equation}
  \sum_{n=2}^\infty c_{n} z^n  =  \sum_{n=1}^\infty c_{n} z^{n+1} +  \sum_{n=0}^\infty c_{n} z^{n+2} +  \sum_{n=2}^\infty 8 z^n
\end{equation}
\begin{handlist}
\item In $ \sum_{n=2}^\infty c_{n-2} z^n$, replace \emph{every} $n$ by $n+2$. That gives
\begin{equation}
  \sum_{n=2}^\infty c_{n-2} z^n =  \sum_{n+2=2}^{n+2 = \infty} c_{n} z^{n+2} = \sum_{n=0}^{\infty} c_{n} z^{n+2}
\end{equation}
\end{handlist}
\end{frame}

\begin{frame}{Step 2}
Use the known values of $c_0$ and $c_1$ to extend the lower sum index of each sum to zero.
\begin{equation}
  \sum_{n=0}^\infty c_{n} z^n  =  z \sum_{n=0}^\infty c_{n} z^{n} +  z^2 \sum_{n=0}^\infty c_{n} z^{n} +  \sum_{n=2}^\infty 8 z^n
\end{equation}
\begin{handlist}
\item If $c_0$ and $c_1$ were nonzero, we'd have a few more terms:
\begin{equation}
  \sum_{n=2}^\infty c_{n} z^n = -c_0 - c_1 z + \sum_{n=0}^\infty c_{n} z^n
\end{equation}
But for our case $c_0 = 0$ and $c_1 = 0$.
\end{handlist}
\end{frame}

\begin{frame}{Step 3}

Define $G(z) =  \sum_{n=0}^\infty c_{n} z^n$ and $F(x) = \sum_{n=2}^\infty 8 z^n$. We have
\begin{equation}
  G(z) = z G(z) + z^2 G(z) + F(z)
\end{equation}
So 
\begin{equation}
  G(z) = \frac{1}{1-z-z^2} F(z).
\end{equation}

\begin{handlist}
\item The function $G$ is the \emph{generating function for the sequence} $c$.

\end{handlist}


\end{frame}
\begin{frame}{Step 4}
From $G$, determine $c_n$. We have
\begin{align*}
 c_n &= \frac{1}{n!}  \left. \D^n_z  \frac{1}{1-z-z^2} F(z) \right  \vert_{z=0}, \\
       &= \frac{1}{n!}  \left. \left. \sum_{k=0}^n \binom{n}{k} \D^k_z \left( \frac{1}{1-z-z^2} \right)  \right \vert_{z=0} \times \D^{n-k} _z F(z )  \right \vert_{z=0} 
\end{align*}

\end{frame}

\begin{frame}{Step 5}
Find the \kth derivative of $\left( \frac{1}{1-z-z^2} \right) $.   The factors of $1-z-z^2$ are messy, so let's just give them names. Say
\begin{align*}
   \left. \D^k_z \left( \frac{1}{1-z-z^2} \right) \right \vert_{z=0} &=  \left. \D^k_z \left( -\frac{1}{(z-a)(z-b)} \right)  \right \vert_{z=0}, \\
                                                                       &=   \frac{k!}{a-b} \left(\frac{1}{a^{k+1}} - \frac{1}{b^{k+1}} \right)
\end{align*}
Actually,
\[
   a =  -\frac{\sqrt{5}+1}{2}, \quad b = \frac{\sqrt{5}-1}{2}
\]
That makes $a-b = -\sqrt{5}$. 
\end{frame}

\begin{frame}{Step 6}
Find the n-k th derivative of $F(z) $.

\begin{equation}
D^{n-k} _z F(z ) \vert_{z=0}  = (n-k) ! \begin{cases} 0 &  n -k < 2 \\  8  & n -k \geq  2 \end{cases}.
\end{equation}
\end{frame}
\begin{frame}{Step 7}
Collecting these results gives

\[
 c_n = -\frac{8}{\sqrt{5}} \sum_{k=0}^{n-2} \left ( \frac{1}{a^{k+1}} - \frac{1}{b^{k+1}} \right) 
\]

\begin{handlist}
\item What's the story with all the factorials and binomial coefficients?
\item  $\frac{1}{n!} \binom{n}{k} k! (n-k)! = 1$, that's what!
\end{handlist}
\end{frame}

\begin{frame}{Step 8}

We could sum these geometric sums 
\[
 c_n = -\frac{8}{\sqrt{5}} \sum_{k=0}^{n-2} \left ( \frac{1}{a^{k+1}} - \frac{1}{b^{k+1}} \right) 
\]
but the result is messy.  So, let's settle for an approximation. The term $b^{-k}$ grows exponentially while 
$a^{-k}$ decays exponentially, so let's ignore all terms except the growing exponential. That gives
\[
 c_n \approx  \frac{8}{\sqrt{5}} \sum_{k=0}^{n-2} \frac{1}{b^{k+1}} \approx  \frac{8 b}{\sqrt{5}} \frac{{{b}^{-n}}}{1-b}
\]
\begin{equation}
   c_n \approx 5.8 \times 1.6^n
\end{equation}
\end{frame}

\begin{frame}
The value of 
\begin{align*}
c_{300} &= 2876634 \dots (\text{lots of digits deleted}) \dots 044718400, \\
             &\approx 5.626 \times 10^{209}.
\end{align*}
Our approximation gives $c_{300} \approx 5.8 \times 1.6^{300}  =  5.633 \times 10^{209}.$

\begin{handlist}
\item At $10^{15}$ FLOPS, it would take about $10^{189}$ years to compute $P_{300}$ using 
the Legendre polynomial recursion without memoization. 

\item But with memoization, computing $P_{300}$ takes a few microseconds.

\end{handlist}
\end{frame}

\begin{frame}{For further study}

\emph{CONCRETE MATHEMATICS: A Foundation for Computer Science}, 2nd Edition, Ronald  Graham, 
Donald Knuth, and Oren Patashnik,
Addison-Wesley Professional, 1994.

\end{frame}

\begin{frame}{What did chatGPT about our class?}

Upon reviewing the logical flow of the work, I didn't find any significant logical errors. However, I'll provide some feedback on specific aspects:

\begin{itemize}
    \item \textbf{Binomial Coefficients:} The explanation and usage of binomial coefficients seem correct.

    \item \textbf{Product Rule for nth Derivatives:}  The expression provided for the nth derivative of the product of two functions is accurate and follows the standard mathematical notation.

    \item \textbf{Derivatives of Power Series:}  The statement regarding the nth derivative of a power series is correct, but it's essential to ensure that the assumptions about the convergence of the series hold true.



\end{itemize}

\end{frame}

\begin{frame}
\begin{itemize}

    \item \textbf{Derivatives of Rational Functions:} The formulas for the nth derivative of a rational function appear to be correctly derived and presented.
    
    \item \textbf{Generating Function and Recursive Sequence:} The steps taken to derive the generating function for the given recursive sequence seem logical and well-justified. The subsequent manipulation of the generating function to find the explicit formula for the sequence also appears to be sound.

\item \textbf{Approximation} The approximation made to simplify the expression seems reasonable, although it's important to note that it introduces an approximation error. The comparison with the exact value of $c_{300}$ helps to assess the accuracy of the approximation. 


\end{itemize}
Overall, the logical progression of the work seems sound, with each step building upon the previous ones in a coherent manner. However, it's always a good practice to double-check the calculations and assumptions to ensure accuracy.

\end{frame}

\begin{frame}{More from chatGPT} 

This lecture seems like a great fit for third-year computer science majors. It covers essential mathematical concepts that are directly relevant to their studies and future careers. By exploring generating functions, binomial coefficients, and derivatives of functions, students will gain valuable insights into algorithmic analysis and data structures.

What's particularly engaging is how the lecture applies these theoretical concepts to a practical problem: solving a recurrence relation. This hands-on approach helps students see the real-world implications of the math they're learning, making the material more relatable and exciting.

The discussion on approximation adds another layer of interest by demonstrating how to balance precision and efficiency in problem-solving. Plus, the comparison between exact and approximate values provides a tangible way for students to gauge the accuracy of their solutions.

Overall, this lecture offers a solid blend of theory and application, making it both informative and engaging for students at this stage of their academic journey.

\end{frame}
\end{document}