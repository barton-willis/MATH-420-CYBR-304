\documentclass[portrait,fleqn,12pt]{beamer}
\usetheme[progressbar=frametitle]{metropolis}
\usepackage{amsmath}
\usepackage{fleqn,amssymb,shadow}
\usepackage{float}
\usepackage{epsfig}
\usepackage{enumerate}
\usepackage{cancel}
\usepackage{fontawesome5}
%\usepackage{fourier}
\usepackage{upgreek}
\raggedright

% Custom commands
\newcommand{\fma}{{\rm fma}}
\newcommand{\reals}{\mathbf{R}}
\newcommand{\epsmach}{\varepsilon_m}
\newcommand{\mymax}{\mbox{max} \,\,}
\newcommand{\mydeg}{\mbox{deg}}
\newcommand{\dom}{\mbox{dom}}
\newcommand{\fl}{\mbox{fl}}
\newcommand{\rel}{\mbox{rel}}
\newcommand{\cond}{\mbox{cond}}
\newcommand{\bern}{\mathcal{B}}
\newcommand{\order}{\mathcal{O}}
\newcommand{\atan}{\rm{arctan}}
\newcommand{\bfa}{\mathbf{a}}
\newcommand{\bfb}{\mathbf{b}}
\newcommand{\bfzero}{\mathbf{0}}
\newcommand{\bfx}{\mathbf{x}}
\newcommand{\complex}{\mathbf{C}}
\newcommand{\integers}{\mathbf{Z}}

% Custom environments
\newenvironment{alphalist}
   {\begin{enumerate}[(a)]
       \addtolength{\itemsep}{0.0\itemsep}}
     {\end{enumerate}}

\newenvironment{handlist}
   {\begin{enumerate}[\faHandPointRight]
       \addtolength{\itemsep}{0.0\itemsep}}
     {\end{enumerate}}

\newenvironment{define}[1]{
  \textbf{Definition} #1}{}

\newenvironment{idefine}[2]{
  \index{#1}
  \textbf{Definition} #2}{\(\blacktriangleleft\)}

\newenvironment{myexample}[1]{
  \textbf{Example} #1}


\newenvironment{mynote}[1]{
  \textbf{Note} #1}

\begin{document}

\begin{frame}
\begin{flushleft} 
\textbf{Can I do X?} \\
MATH 202 \\
\today 
\end{flushleft}


\emph{“The law is reason unaffected by desire.”} \hfill {\sc Aristotle } 
\end{frame}


\begin{frame}[fragile]{Our problem}
We would like to find an explicit formula for the sequence $c$ that is defined recursively by
\begin{equation}
  c_n= \begin{cases} 0 & n=0  \text{ or } n = 1 \\
                                 c_{n-1} + c_{n-2}+ 8 & n \in \integers_{\geq 2}
           \end{cases}
\end{equation}
\begin{handlist}
\item The first eleven terms of the seqence $c$ are
\begin{equation}
      0,0,8,16,32,56,96,160,264,432,704
\end{equation}
\item To complete this task, we need three new tools
\end{handlist}
\end{frame}
\begin{frame}{Tool 1:  Binomial Coefficents}

For positive integers $n$ and $k$ with $n \geq k$, we define
\begin{equation}
      \binom{n}{k} = \frac{n!}{k! (n-k)!}
\end{equation}
\begin{handlist}
 \item Spoken $\binom{n}{k}$ is ``$n$ choose $k$.''
 \item For a finite set $A$ with $n$ members,  $\binom{n}{k}$ is the number of subsets of $A$ that have $k$ members.
\end{handlist}
\end{frame}

\begin{frame}{Tool 2:  Product rule for $n$-th derivatives}

For smooth functions $f$ and $g$ and a nonnegative integer $n$, the n-th derivative of the product of $f$ with $g$ is
\begin{equation}
       D^n (f g) = \sum_{k=0}^n  \binom{n}{k}  f^{k} g^{n-k}.
\end{equation}
\end{frame}

\begin{frame}{Tool 3:  $n$-th derivative of a power series}

Let $c$ be a sequnce and let $n$ be a nonnegative integer.  Then
\begin{equation}
       D^n_x \sum_{k=0}^\infty c_k x^k  \vert_{x=0} = n!  c_n.
\end{equation}
\begin{handlist}
   \item For this to be true, it has to be the case that  the radius of convergence of the series is nonzero.
\end{handlist}
\end{frame}

\begin{frame}{Tool 4:   $n$-th derivative of a rational function}

Let $a \in \reals$ and $n$ a nonnegative integer. Then
\begin{equation}
       D^n_x \left(\frac{1}{x-a}  \right) \vert_{x=0} = -\frac{n!}{a^{n+1}}.
\end{equation}
\end{frame}

\begin{frame}{Step 0}

Multiply the recursion $c_n = c_{n-1} + c_{n-2}+ 8$ by $z^n$ and sum from $n=2$ to $\infty$.  This gives
\begin{equation}
  \sum_{n=2}^\infty c_{n} z^n  =  \sum_{n=2}^\infty c_{n-1} z^n +  \sum_{n=2}^\infty c_{n-2} z^n +  \sum_{n=2}^\infty 8 z^n
\end{equation}
\begin{handlist}
\item The lowest sum index is two because the recursion  $c_n = c_{n-1} + c_{n-2}+ 8$ is only valid for $n \geq 2$.
\end{handlist}
\end{frame}
\begin{frame}{Step 1}

If needed shift each sum index to make the each summand involve $c_k$.
\begin{equation}
  \sum_{n=2}^\infty c_{n} z^n  =  \sum_{n=1}^\infty c_{n} z^{n+1} +  \sum_{n=0}^\infty c_{n} z^{n+2} +  \sum_{n=2}^\infty 8 z^n
\end{equation}
\begin{handlist}
\item The lowest sum index is two because the recursion  $c_n = c_{n-1} + c_{n-2}+ 8$ is only valid for $n \geq 2$.
\end{handlist}
\end{frame}

\begin{frame}{Step 2}
Use the know values of $c_0$ and $c_1$ to extend the lower sum index of each sum to zero.
\begin{equation}
  \sum_{n=0}^\infty c_{n} z^n  =  z \sum_{n=0}^\infty c_{n} z^{n} +  z^2 \sum_{n=0}^\infty c_{n} z^{n} +  \sum_{n=2}^\infty 8 z^n
\end{equation}
\begin{handlist}
\item If $c_0$ and $c_1$ were nonzero, we'd have a few more terms!
\end{handlist}
\end{frame}

\begin{frame}{Step 3}

Define $G(z) =  \sum_{n=0}^\infty c_{n} z^n$ and $F(x) = \sum_{n=2}^\infty 8 z^n$. We have
\begin{equation}
  G(z) = z G(z) + z^2 G(z) + F(z)
\end{equation}
So 
\begin{equation}
  G(z) = \frac{1}{1-z-z^2} F(x).
\end{equation}



\end{frame}
\begin{frame}{Step 3}
From $G$, determine $c_k$. We have
\begin{align*}
 c_n &= \frac{1}{n!} D^n_z  \frac{1}{1-z-z^2} F(x) \vert_{x=0}, \\
       &= \frac{1}{n!}  \sum_{k=0}^n \binom{n}{k} D^k_z \left( \frac{1}{1-z-z^2} \right)  D^{n-k} _z F(z ) \vert_{z=0} 
       &= 
\end{align*}

\end{frame}

\end{document}

