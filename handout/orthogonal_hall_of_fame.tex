\documentclass[portrait,fleqn,12pt]{beamer}
\usetheme[progressbar=frametitle]{metropolis}
\usepackage{amsmath}
\usepackage{fleqn,amssymb,shadow}
\usepackage{float}
\usepackage{epsfig}
\usepackage{enumerate}
\usepackage{cancel}
\usepackage{fontawesome5}
%\usepackage{fourier}
\usepackage{upgreek}
%\raggedright

% Custom commands
\newcommand{\fma}{{\rm fma}}
\newcommand{\reals}{\mathbf{R}}
\newcommand{\epsmach}{\varepsilon_m}
\newcommand{\mymax}{\mbox{max} \,\,}
\newcommand{\mydeg}{\mbox{deg}}
\newcommand{\dom}{\mbox{dom}}
\newcommand{\fl}{\mbox{fl}}
\newcommand{\rel}{\mbox{rel}}
\newcommand{\cond}{\mbox{cond}}
\newcommand{\bern}{\mathcal{B}}
\newcommand{\order}{\mathcal{O}}
\newcommand{\atan}{\rm{arctan}}
\newcommand{\bfa}{\mathbf{a}}
\newcommand{\bfb}{\mathbf{b}}
\newcommand{\bfzero}{\mathbf{0}}
\newcommand{\bfx}{\mathbf{x}}
\newcommand{\complex}{\mathbf{C}}
\newcommand{\integers}{\mathbf{Z}}
\newcommand{\D}{\mathrm{D}}

% Custom environments
\newenvironment{alphalist}
   {\begin{enumerate}[(a)]
       \addtolength{\itemsep}{0.0\itemsep}}
     {\end{enumerate}}

\newenvironment{handlist}
   {\begin{enumerate}[\faHandPointRight]
       \addtolength{\itemsep}{0.0\itemsep}}
     {\end{enumerate}}

\newenvironment{define}[1]{
  \textbf{Definition} #1}{}

\newenvironment{idefine}[2]{
  \index{#1}
  \textbf{Definition} #2}{\(\blacktriangleleft\)}

\newenvironment{myexample}[1]{
  \textbf{Example} #1}


\newenvironment{mynote}[1]{
  \textbf{Note} #1}

\begin{document}

\begin{frame}
\begin{flushleft} 
\textbf{Fun with generating functions} \\
MATH 420 \& CYBR 304 \\
Spring 2024 
\end{flushleft}



\end{frame}

\begin{frame}{Long ago, we defined}
\begin{define}
Let $\{f_1, f_2, \dots, f_n \}  \subset  C_{[a,b]}$ and let $\langle \cdot, \cdot \rangle$ be an inner product
on $C_{[a,b]}$.  The set of functions $\{f_1, f_2, \dots, f_n \}$ is \emph{orthogonal} provided distinct pairs
of these functions are orthogonal.  The set is \emph{orthonormal} provided
\begin{equation}
\langle f_k, f_\ell \rangle = \begin{cases} 1 & k=\ell \\ 0 & k \neq \ell \end{cases}
\end{equation}
\end{define}

\begin{handlist}
\item Every orthonormal set is orthogonal.

\item When $\{f_1, f_2, \dots, f_n \} $ is orthogonal, but not orthonormal, we can define
\begin{equation}
   \widehat f_k = \frac{f_k}{\sqrt{ \langle f_k, f_k \rangle}}.
\end{equation}
Then the set $\{\widehat f_1, \widehat f_2, \dots, \widehat f_n \}$ is orthonormal.
\end{handlist}
\end{frame}

\begin{frame}{More facts from the past}

Let  $\{f_1, f_2, \dots, f_n \} \subset C_{[a,b]}$, and suppose that this set is orthnormal. Given $F \in C_{[a,b]}$,
we would like to find $c_1, c_2, \dots c_n$ that minimize the function
\begin{equation}
  (c_1, c_1, \dots, c_n) \in \reals^n \mapsto \| F - \sum_{k=0}^n c_k f_k, F  \|_2^2
 \end{equation}
The solution is
\begin{equation}
   c_k  =  \langle F, f_k \rangle, \text{ for } k \in 1 \dots n.
\end{equation}
\begin{handlist}
\item Recall $\| f \|_2^2 = \langle f, f \rangle$
\end{handlist}
\end{frame}

\begin{frame}{A Hall of Fame Orthogonal set}

\textbf{Fact} For $k, \ell \in \integers$, we have
\begin{align*}
&\int_{-\uppi}^\uppi \cos(k x) \cos(\ell x) \, \mathrm{d} x = \begin{cases} \uppi & k = \ell \\ 0  & k \neq  \ell \end{cases},\\
&\int_{-\uppi}^\uppi \cos(k x) \sin(\ell x) \, \mathrm{d} x = 0, \\
&\int_{-\uppi}^\uppi \sin(k x) \sin(\ell x) \, \mathrm{d} x = \begin{cases} \uppi & k = \ell \\ 0  & k \neq  \ell \end{cases}.
\end{align*}


\end{frame}
\end{document}