\documentclass[usenames,dvipsnames,fleqn]{beamer}
%\usetheme[height=7mm]{Rochester}
\usetheme{Boadilla}

\setbeamertemplate{caption}[numbered]
\setbeamertemplate{footline}[text line]{%
\parbox{\linewidth}{\vspace*{-8pt}\hfill\insertshortauthor\hfill\insertpagenumber}}
\setbeamertemplate{navigation symbols}{}
\author[BW]{BW}


\usepackage{amsmath}
\usepackage{amsthm}
%\usepackage{isomath}
\usepackage{comment}
%\usepackage[dvipsnames]{color}
%\usefonttheme{professionalfonts}
%\usefonttheme{serif}

\newcommand{\reals}{\mathbf{R}}
\newcommand{\complex}{\mathbf{C}}
\newcommand{\integers}{\mathbf{Z}}
\DeclareMathOperator{\range}{range}
\DeclareMathOperator{\cond}{cond}
\DeclareMathOperator{\domain}{domain}
\DeclareMathOperator{\codomain}{codomain}
\DeclareMathOperator{\sspan}{span}
\DeclareMathOperator{\F}{F}
\DeclareMathOperator{\G}{G}
\DeclareMathOperator{\B}{B}

\DeclareMathOperator{\fl}{fl}

\DeclareMathOperator{\nullspace}{nullity}
\theoremstyle{definition}
\newtheorem{mydef}{Definition}
\newtheorem{myex}{Example}
\newtheorem{myth}{Theorem} 
\newtheorem{myfact}{Fact}
\newtheorem{metathm}{Meta Theorem}
\newtheorem{Question}{Question}
\newtheorem{Answer}{Qnswer}
\newtheorem{myproof}{Proof}
\newtheorem{hurestic}{Hurestic}

\usepackage[english]{babel}
\usepackage[final]{microtype}
\usepackage[]{xcolor}
%\usefonttheme{professionalfonts}
%\usefonttheme{serif}
\usepackage{courier}
%\lstset{basicstyle=\ttfamily\footnotesize,breaklines=true}
%\lstset{framextopmargin=50pt,frame=bottomline}
%\usepackage{lipsum}  % generates filler text


%\begin{document}

\usepackage[]{graphicx}

\newcommand{\foo}{\hspace{-2.3pt}$\bullet$ \hspace{5pt}}

%--------
%usepackage[usenames,dvipsnames,svgnames,table]{color}

\usepackage{comment}
\usepackage{wrapfig}
\begin{document}


\begin{frame}{Small change, big effect}


Small changes in inputs can cause big changes in the output. For example, the solution set to 
\begin{align*}
    1000 x - 1000 y &= -1000\\
   1001 x - 1000 y &= -1000
\end{align*}
is \(x = 0, y = 1 \).  But the solution set to
\begin{align*}
    1000 x - 1000 y &= -1000 \\
   1001 x - 1000 y &= -900
\end{align*}
is \(x=100,y=101\). 


\begin{enumerate}

\item  Decreasing the constant term in the second equation by 10\% from -1000 to  \(-900\), changes the solution set from \(x=0,y=1\) to 
 \(x=100,y=101\). 
 
 \item It's the butterfly effect.
\end{enumerate}

\end{frame}

\begin{frame}
In matrix form, these two sets of linear equations are
\begin{align*}
 \begin{bmatrix} 1000& -1000 \\ 1001 & -1000 \end{bmatrix} \begin{bmatrix} x \\ y \end{bmatrix} &= 
     \begin{bmatrix} -1000 \\ -1000 \end{bmatrix}, \\
 \begin{bmatrix} 1000 & -1000 \\ 1001 & -1000 \end{bmatrix} \begin{bmatrix} x \\ y \end{bmatrix} &= 
     \begin{bmatrix} -1000 \\ -900 \end{bmatrix}.   
\end{align*}

\begin{enumerate}

\item These linear equations have the same coefficient matrix and the constant terms differ by at most 10\%.

\item But the solution sets are far apart. 

\item The determinate of each coefficient matrix \(1000\).

\item a  determinant near zero \emph{isn't} required for the butterfly effect.

\item The solution set is the intersection of two lines that are nearly parallel.

\item Something like nearly parallel lines \emph{is }  required for the butterfly effect.

\end{enumerate}
\end{frame}

\begin{frame}
Consider the linear equations
\[
   M x = y, \quad \widehat M  \, \widehat x = \widehat y.
\]

\begin{enumerate}

\item Unhatted variables are the true value, and hatted variables are approximations to the true value.

\item We might have, for example,
\[
   M = \begin{bmatrix} \pi & \sqrt{3} \\ \frac{1}{10} & 23 \end{bmatrix},  
   \widehat M =   \begin{bmatrix} \fl(\pi) & \fl(\sqrt{3}) \\ \fl(\frac{1}{10}) & \fl(23) \end{bmatrix}
   y = \begin{bmatrix} \frac{2}{3} \\ \frac{1}{42} \end{bmatrix}, 
   \widehat y = \begin{bmatrix} \fl(\frac{2}{3}) \\ \fl(\frac{1}{42}) \end{bmatrix} 
   \]
where \(\fl\) rounds to the nearest float.

\item We would like to bound \( \| x - \widehat x \| \).

\item Let's not commit to a specfic norm.

\item We'll assume that both \(M\) and \(\widehat M \) have inverses.
\end{enumerate}

\end{frame}

\begin{frame}

We know something about \(\| y - \widehat y\| \) and  \( \| M  - \widehat M \| \), but we know nothing directly about
\(M^{-1} y - \widehat M^{-1} \widehat y\); thus
\begin{align*}
   x - \widehat x &= M^{-1} y - \widehat M^{-1} \widehat y, \\
       &=M^{-1} \left(y - \widehat y + \widehat y \right)  - \widehat M^{-1} \widehat y, &(\mbox{add and subtract}), \\
       &= M^{-1} \left(y - \widehat y \right) + \left(M^{-1} - \widehat M^{-1} \right) \widehat y.  &(\mbox{regroup})\\
\end{align*}
So
\begin{align*}
  \|  x - \widehat x \| &= \|M^{-1} \left(y - \widehat y \right) + \left(M^{-1} - \widehat M^{-1} \right) \widehat y \|,\\
       &\leq  \|M^{-1} \left(y - \widehat y \right) \| + \| \left(M^{-1} - \widehat M^{-1} \right) \widehat y \|. &(\mbox{triangle inquality})
\end{align*}
We'll bound each term of the last two terms separately.

\end{frame}
\begin{frame}

\begin{align*}
   \|M^{-1} \left(y - \widehat y \right) \| &\leq \| M^{-1} \| \| y - \widehat y  \|,  &(\mbox{basic norm fact}) \\
               &\leq  \varepsilon_m \| M^{-1} \| y  \|, &(\mbox{assumption}) \\
               &= \varepsilon_m \| M^{-1} \| M x  \|,  &(\mbox{defintion of } x ), \\
               &=  \varepsilon_m \| M^{-1} \|  \| M  \| \|x  \|.  &(\mbox{basic norm fact}) 
\end{align*}
\end{frame}

\begin{frame}

Bounding the other term \( \left(M^{-1} - \widehat M^{-1} \right) \widehat y \) is tricky. A good warmup 
is a simple algebra fact:

\textbf{Warmup}  For \(a,b \in \reals_{\neq 0}\), we have
\[
   a^{-1} - b^{-1} = \frac{1}{a} - \frac{1}{b} = \frac{b-a}{a b} = a^{-1} (b - a) b^{-1}
\]


\textbf{Fact} If \(a\) and \(b\) are invertible matrices, then
\[
    a^{-1} - b^{-1} = a^{-1} (b - a) b^{-1}.
\]
But for matrices \(a\) and \(b\), the expression  \(\frac{b-a}{a b}\) is ambiguoyus .


\end{frame}
\begin{frame}
 \begin{align*}
    \left(M^{-1} - \widehat M^{-1} \right) \widehat y &= M^{-1} (\widehat M - M) \widehat M^{-1} \widehat y, \\
    &=   M^{-1} (\widehat M - M) \widehat x.
 \end{align*}
 The norm is bounded by
 \begin{align*}
  \| \left(M^{-1} - \widehat M^{-1} \right) \widehat y \| &= \| M^{-1} \| 
     \|  \widehat M - M \| \|\widehat x \|, \\
     &\leq  \varepsilon_m  \| M^{-1} \|  \| M \|  \|\widehat x \|.
 \end{align*} 
 The product  \(\| M^{-1} \|  \| M \| \) appears in the upper bounds for both of our terms. Let's name it:
 \[
    \cond_M = \| M^{-1} \|  \| M \|.
 \]
 We've shown that 
 \[
    \| x - \widehat x \| \leq \varepsilon_m \cond_M \left (\| x \| + \| \widehat x \| \right).
 \]
\end{frame}
\begin{frame}
Final refinement:
\begin{align*} 
    \| x - \widehat x \| & \leq \varepsilon_m \cond_M \left (\| x \| + \| \widehat x - x + x \| \right), \\
                &\leq  \varepsilon_m \cond_M \left (\| x \| + \| \widehat x - x  \| + \| x \| \right).
 \end{align*}
 Combining the \(\| x - \widehat x \| \) terms on the left and the right and assuming 
 \(1 - \varepsilon_m \cond_M   > \) gives
 \[
     \| x - \widehat x \| \leq \varepsilon \frac{2 \cond_M}{1 - \varepsilon_m \cond_M } \| x \| .
 \]
\end{frame}
\end{document}
\begin{frame}[fragile]{Famous matrix reduction}

Step \#1, eliminate in the first column
\[
  \begin{bmatrix} 1 & 2 & 3 \\ 4 & 5 & 6 \\ 7 & 8 & 9 \end{bmatrix} 
     \begin{matrix}  \phantom{1}  \\ R_2 \leftarrow R_2 - 4 R_1  \\ R_3 \leftarrow R_3 - 7 R_1  \end{matrix}
     \begin{bmatrix} 1 & 2 & 3 \\ 0 & -3 & -6 \\ 0 & -6 & -12 \end{bmatrix}
\]
\begin{enumerate}
 \item The 1,1 entry is nonzero, so it can be used as a piviot.
 \item Each elimination looks like \(R_k \leftarrow R_k + \mbox{ multiplier } \times  R_1 \).
 \item Each multipler is a quotient with the pivot in the denominator.
\end{enumerate}

\end{frame}
\begin{frame}[fragile]
Step \#2, eliminate in the second column
\[
   \begin{bmatrix} 1 & 2 & 3 \\ 0 & -3 & -6 \\ 0 & -6 & -12 \end{bmatrix}
     \begin{matrix}  \phantom{1}  \\ \phantom{1}  \\ R_3 \leftarrow R_3 - 2 R_2  \end{matrix}
     \begin{bmatrix} 1 & 2 & 3 \\ 0 & -3 & -6 \\ 0 & 0  & 0 \end{bmatrix}
\]
\begin{enumerate}
 \item The 2,2 entry is nonzero, so it can be used as a piviot.
 \item We're done eliminating--the result is an upper triangular matrix.
\end{enumerate}
\end{frame}
\begin{frame}[fragile]{Famous matrix reduction redux}


\begin{enumerate}
 \item Each step in row reduction can be done by left multiplication by a lower triangular matrix.

\end{enumerate}
Step \#1, eliminate in the first column
\[
  \underbrace{\begin{bmatrix} 1 & 0 & 0 \\ -4 & 1 & 0 \\ -7 & 0 & 1 \end{bmatrix} }_{=E_1}
  \underbrace{\begin{bmatrix} 1 & 2 & 3 \\ 4 & 5 & 6 \\ 7 & 8 & 9 \end{bmatrix} }_{=M}
  =    \begin{bmatrix} 1 & 2 & 3 \\ 0 & -3 & -6 \\ 0 & -6 & -12 \end{bmatrix}
\]

\end{frame}

\begin{frame}[fragile]
Step \#2, eliminate in the second column
\[
  \underbrace{\begin{bmatrix} 1 & 0 & 0 \\ 0 & 1 & 0 \\ 0 & -2 & 1 \end{bmatrix}}_{=E_2}
  \begin{bmatrix} 1 & 2 & 3 \\ 0 & -3 & -6 \\ 0 & -6 & -12 \end{bmatrix} =
   \underbrace{ \begin{bmatrix} 1 & 2 & 3 \\ 0 & -3 & -6 \\ 0 & 0  & 0 \end{bmatrix}}_{=U}
\]

Symbolically, this is \(E_2 E_1 M = U \).


\end{frame}
\begin{frame}[fragile]
\textbf{Question:} How to undo \(R_2 \leftarrow R_2 - 4 R_1 \) and \( R_3 \leftarrow R_3 - 7 R_1\)? 

\textbf{Answer:} Do  \(R_2 \leftarrow R_2 + 4 R_1 \) and \( R_3 \leftarrow R_3 + 7 R_1\).

\vspace{0.1in}

\textbf{Question} What is the matrix associated with \(R_2 \leftarrow R_2 + 4 R_1,  \,  R_3 \leftarrow R_3 + 7 R_1\)? 

\textbf{Answer} It is 
\(
  \begin{bmatrix} 1 & 0 & 0 \\ 4 & 1 & 0 \\ 7 & 0 & 1  \end{bmatrix} 
\)

Check:
\[
  \begin{bmatrix} 1 & 0 & 0 \\ 4 & 1 & 0 \\ 7 & 0 & 1 \end{bmatrix}
   \begin{bmatrix} 1 & 0 & 0 \\ -4 & 1 & 0 \\ -7 & 0 & 1 \end{bmatrix} =
    \begin{bmatrix} 1 & 0 & 0 \\ 1 & 0 & 0 \\ 1 & 0 & 1 \end{bmatrix} = I
  \]


\end{frame}

\begin{frame}[fragile]
Multipling \(E_2 E_1 M = U\) on the left by \(E_1^{-1} E_2^{-1} \) gives
 \[
 	    E_1^{-1} \underbrace{E_2^{-1} E_2}_{ = I} E_1 M = \underbrace{E_1^{-1} E_1}_{=I}  M =  M =  E_1^{-1} E_2^{-1} U
 \] 
 Let's look at \( E_1^{-1} E_2^{-1} \):
 \[
    \begin{bmatrix} 1 & 0 & 0 \\ 4 & 1 & 0 \\ 7 & 0 & 1 \end{bmatrix} 
      \begin{bmatrix} 1 & 0 & 0 \\ 0 & 1 & 0 \\ 0 & 2 & 1 \end{bmatrix} 
      = \begin{bmatrix} 1 & 0 & 0 \\ 4 & 1 & 0 \\ 7 & 2 & 1 \end{bmatrix}
 \]
 \begin{enumerate}
  \item The entries of  \(E_1^{-1} E_2^{-1}  \) below the diagonal are the negation of  multipliers; the diagonal entries are
  1s.
  \item This isn't a coincidence.
  
  \item We've factored a matrix into a product of a lower triangular matrix with 1s on the diagonal times an upper triangular matrix.
  
  \item This is the Doolittle factoriation. 
 \end{enumerate}
 
\end{frame}

\begin{frame}[fragile]
Specfically, the Doolittle factoratation of our famuous matrix is
\[
     \begin{bmatrix} 1 & 2 & 3 \\ 4 & 5 & 6 \\ 7 & 8 & 9 \end{bmatrix}  = 
        \begin{bmatrix} 1 & 0 & 0 \\ 4 & 1 & 0 \\ 7 & 2 & 1 \end{bmatrix} 
           \begin{bmatrix} 1 & 2 & 3 \\ 0 & -3 & -6 \\ 0 & 0  & 0 \end{bmatrix}
\]

\textbf{Fact:} When a pivot is zero, rows need to be swapped to place a nonzero entry on the pivot. This results in a factorization
\[
    P M = L U,
\]
where \(P\) is a matrix formed by swapping rows of \(I\), the matrix \(L\) is lower triangular, and \(U\) is upper triangular.

\end{frame}
\begin{frame}[fragile]{Maxima LU}

\begin{enumerate}

\item Maxima's \texttt{lu\_factor} returns the lower triangular factor in in the lower part of the matrix, and the upper triangular part in the same matrix.
 The 1s on the diagonal are implied.

\item The second item returned \texttt{lu\_factor} tells us if there were row swaps. 

\item The third item return tells us what arithematic Maxima used.

\item The \texttt{get\_lu\_factors} returns the matrices \(P, L, U\).
\end{enumerate}
\noindent
%%%%%%%%%%%%%%%
%%% INPUT:
\begin{minipage}[t]{8ex}{\color{red}\bf
\begin{verbatim}
(%i4) 
\end{verbatim}
}%
\end{minipage}%
\begin{minipage}[t]{\textwidth}{\color{blue}
\begin{verbatim}
lu_factor(matrix([1,2,3],[4,5,6],[7,8,9]));
\end{verbatim}
}%
\end{minipage}%
%%% OUTPUT:
\definecolor{labelcolor}{RGB}{100,0,0}


\begin{math}\displaystyle
\parbox{8ex}{\color{labelcolor}(\%o4) }
[\begin{pmatrix}1 & 2 & 3\cr 4 & -3 & -6\cr 7 & 2 & 0\end{pmatrix},[1,2,3],generalring]\mbox{}
\end{math}
%%%%%%%%%%%%%%%

\noindent
%%%%%%%%%%%%%%%
%%% INPUT:
\begin{minipage}[t]{8ex}{\color{red}\bf
\begin{verbatim}
(%i5) 
\end{verbatim}
}%
\end{minipage}%
\begin{minipage}[t]{\textwidth}{\color{blue}
\begin{verbatim}
get_lu_factors(%);
\end{verbatim}
}%
\end{minipage}%
%%% OUTPUT:
\definecolor{labelcolor}{RGB}{100,0,0}


\begin{math}\displaystyle
\parbox{8ex}{\color{labelcolor}(\%o5) }
[\begin{pmatrix}1 & 0 & 0\cr 0 & 1 & 0\cr 0 & 0 & 1\end{pmatrix},\begin{pmatrix}1 & 0 & 0\cr 4 & 1 & 0\cr 7 & 2 & 1\end{pmatrix},\begin{pmatrix}1 & 2 & 3\cr 0 & -3 & -6\cr 0 & 0 & 0\end{pmatrix}]\mbox{}
\end{math}
%%%%%%%%%%%%%%%

\noindent
%%%%%%%%%%%%%%%
%%% INPUT:
\begin{minipage}[t]{8ex}{\color{red}\bf
\begin{verbatim}
(%i6) 
\end{verbatim}
}%
\end{minipage}%
\begin{minipage}[t]{\textwidth}{\color{blue}
\begin{verbatim}
first(%) . second(%) . third(%);
\end{verbatim}
}%
\end{minipage}%
%%% OUTPUT:
\definecolor{labelcolor}{RGB}{100,0,0}


\begin{math}\displaystyle
\parbox{8ex}{\color{labelcolor}(\%o6) }
\begin{pmatrix}1 & 2 & 3\cr 4 & 5 & 6\cr 7 & 8 & 9\end{pmatrix}\mbox{}
\end{math}
%%%%%%%%%%%%%%%

\end{frame}


\end{document}

