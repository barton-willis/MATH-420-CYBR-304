\documentclass[portrait,fleqn,12pt]{beamer}
\usetheme[progressbar=frametitle]{metropolis}
\usepackage{amsmath}
\usepackage{fleqn,amssymb,shadow}
\usepackage{float}
\usepackage{epsfig}
\usepackage{enumerate}
\usepackage{cancel}
\usepackage{fontawesome5}
%\usepackage{fourier}
\usepackage{upgreek}
%\raggedright

% Custom commands
\newcommand{\fma}{{\rm fma}}
\newcommand{\reals}{\mathbf{R}}
\newcommand{\epsmach}{\varepsilon_m}
\newcommand{\mymax}{\mbox{max} \,\,}
\newcommand{\mydeg}{\mbox{deg}}
\newcommand{\dom}{\mbox{dom}}
\newcommand{\fl}{\mbox{fl}}
\newcommand{\rel}{\mbox{rel}}
\newcommand{\cond}{\mbox{cond}}
\newcommand{\bern}{\mathcal{B}}
\newcommand{\order}{\mathcal{O}}
\newcommand{\atan}{\rm{arctan}}
\newcommand{\bfa}{\mathbf{a}}
\newcommand{\bfb}{\mathbf{b}}
\newcommand{\bfzero}{\mathbf{0}}
\newcommand{\bfx}{\mathbf{x}}
\newcommand{\complex}{\mathbf{C}}
\newcommand{\integers}{\mathbf{Z}}
\newcommand{\D}{\mathrm{D}}
\newcommand{\cis}{\mathrm{cis}}
\newcommand{\euler}{\mathrm{e}}
\newcommand{\imag}{\mathrm{i}}
\newcommand{\nth}{n\textsuperscript{th}\,}
\newcommand{\kth}{k\textsuperscript{th}\,}

\newcommand{\re}{\mathrm{Re}}
% Custom environments
\newenvironment{alphalist}
   {\begin{enumerate}[(a)]
       \addtolength{\itemsep}{0.0\itemsep}}
     {\end{enumerate}}

\newenvironment{handlist}
   {\begin{enumerate}[\faHandPointRight]
       \addtolength{\itemsep}{0.0\itemsep}}
     {\end{enumerate}}

\newenvironment{define}[1]{
  \textbf{Definition} #1}{}

\newenvironment{idefine}[2]{
  \index{#1}
  \textbf{Definition} #2}{\(\blacktriangleleft\)}

\newenvironment{myexample}[1]{
  \textbf{Example} #1}


\newenvironment{mynote}[1]{
  \textbf{Note} #1}

\begin{document}

\begin{frame}
\begin{flushleft} 
\textbf{The Discrete Fourier Transform} \\
MATH 420 \& CYBR 304 \\
Spring 2024 
\end{flushleft}

\emph{“The profound study of nature is the most fertile source of mathematical discovery.”} \hfill{\sc  Joseph Fourier}

\end{frame}


\begin{frame}{A trigonometric basis}  


Let $n \in \integers_{\geq 0}$ and let $a_0$ through $a_n$ and $b_1$ through $b_n$ be real numbers.  Define a function $F$ by 
\begin{equation*}
F(x) = a_0 + \sum_{k=1}^n a_k \cos(k x) +  b_k \sin(k x).
\end{equation*}

Expressing $\cos$ and $\sin$ in exponential form, the formula is 
\begin{align*}
F(x) &= a_0 + \sum_{k=1}^n a_k  \frac{\cis(k x) +  \overline{\cis(k x)}}{2} + b_k  \frac{\cis(k x) -  \overline{\cis(k x)}}{2 \imag}, \\
       & = a_0 + \sum_{k=1}^n \frac{1}{2} (a_k  - \imag b_k) \cis(k x) +   \frac{1}{2} (a_k  + \imag b_k) \overline{\cis(k x) } .  \\
\end{align*}
\end{frame}

\begin{frame}{An exponential basis}
So
\begin{align*}
F(x)   & = a_0 + \sum_{k=1}^n \frac{1}{2} (a_k  - \imag b_k) \cis(k x) +   \frac{1}{2} (a_k  + \imag b_k) \overline{\cis(k x) },   \\
          &= a_0 + \re  \left ( \sum_{k=1}^n (a_k  - \imag b_k) \cis(k x) \right).
\end{align*}
Defining complex numbers $c_0, c_1, \dots, c_n $ as
\begin{equation*}
  c_k = \begin{cases} a_0  & k =0 \\ a_k  - \imag b_k & k \neq 0 \end{cases},
\end{equation*}
we have
\begin{equation*}
F(x)    = \re  \left ( \sum_{k=0}^n c_k \,  \, \cis(k x) \right).
\end{equation*}
\end{frame}

\begin{frame}{Unreal functions}

If we drop the condition that $F$ is real-valued and use the fact that $\overline{\cis(k x) } = \cis(-kx)$ we have
\begin{align*}
F(x)    &= a_0 + \sum_{k=1}^n \frac{1}{2} (a_k  - \imag b_k) \cis(k x) +   \frac{1}{2} (a_k  + \imag b_k) \cis(-k x) ,   \\
         &= \sum_{k=-n}^n c_k \cis(k x).
\end{align*}
where
\begin{equation*}
  c_k = \begin{cases} \frac{1}{2} (a_k  + \imag b_k) & k < 0 \\
                                       a_0 & k = 0 \\
                                       \frac{1}{2} (a_k  - \imag b_k) & k > 0
       \end{cases}.
\end{equation*}
\end{frame}

\begin{frame}{Unreal but orthogonal}

For any $n \in \integers_{\geq 0}$,  the set of functions
\begin{equation*}
  \{ x \in [0, 2 \uppi]  \mapsto \cis(k x) \mid k \in -n \dots n \}
\end{equation*}
is orthogonal with respect to the inner product
\begin{equation*}
  \langle f, g  \rangle = \int_0^{2 \uppi} \overline{f(x)} g(x) \ \mathrm{d} x.
\end{equation*}
In particular for $k, \ell \in \integers$, we have
\begin{equation*}
 \int_0^{2 \uppi}  \overline{ \cis(k x) } \, \cis(\ell x) \mathrm{d} x = \begin{cases} 0 & k \neq \ell \\ 2 \uppi & k = \ell \end{cases}
     = 2 \uppi \delta_{k, \ell}.
 \end{equation*}
\end{frame}

\begin{frame}{The  Kronecker delta function}

We define the Kronecker delta function $\delta$ as
\begin{equation*}
   \delta_{k, \ell} = \begin{cases} 0 & k \neq \ell \\ 1 & k= \ell \end{cases}.
\end{equation*}

\begin{handlist}
\item The arguments to the Kronecker delta function are almost expressed as subscripts.

\item Example $\delta_{\uppi, 3} = 0$ and $\delta_{\uppi, \uppi} = 1$.

\item We have $\sum_{k=0}^n f_k \delta_{k\ell} = \begin{cases} f_\ell &  \ell \in 0 \dots n \\ 0 & \ell \notin 0 \dots n  \end{cases}$.
\end{handlist}
\end{frame}


\begin{frame}{A trigonometric puzzle}

\textbf{Problem} Let $n \in \integers_{\geq 0}$ and let $c_{-n}$ through $c_n$ be complex numbers. Define a 
function $F$ whose formula is 
\begin{equation*}
    F(x) = \sum_{k=-n}^n c_k \euler^{\imag  k x}
\end{equation*}
For a given function $F$, we would like to find a way to find the complex numbers $c_{-n}$ through $c_n$.

\end{frame}

\begin{frame}{Step \#1}
\textbf{Step \#1} Multiply $\displaystyle F(x) = \sum_{k=-n}^n c_k \euler^{\imag  k x}$  by $\euler^{-\imag  \ell x}$. This 
gives
\begin{equation}
    \euler^{-\imag  \ell x} F(x) = \sum_{k=-n}^n c_k 
    \euler^{-\imag  \ell x} \euler^{\imag  k x}
\end{equation}
Using a rule of exponents, we have
\begin{equation}
    \euler^{-\imag  \ell x} F(x) = \sum_{k=-n}^n c_k 
    \euler^{-\imag  (k - \ell) x}.
\end{equation}
\end{frame}

\begin{frame}{Step \#2}
    \textbf{Step \#2} Integrate with respect to $x$ over the 
    interval $[0, 2 \uppi]$
    \begin{equation}
       \int_0^{2 \uppi} \euler^{-\imag  \ell x} F(x) \,
         \mathrm{d}x =  \int_0^{2 \uppi} \sum_{k=-n}^n c_k 
        \euler^{-\imag  (k - \ell) x} \,
        \mathrm{d}x.
 \end{equation}
 Swap the integration and the finite sum:
 \begin{equation}
       \int_0^{2 \uppi} \euler^{-\imag  \ell x} F(x) \,
         \mathrm{d}x =  \sum_{k=-n}^n c_k 
         \int_0^{2 \uppi}  \euler^{-\imag  (k - \ell) x} \,   \mathrm{d}x.
 \end{equation}
 \end{frame}
 
 \begin{frame}{Step \#3}
   \textbf{Step \#3} Integrate the orthogonal functions:
   \begin{equation}
       \int_0^{2 \uppi} \euler^{-\imag  \ell x} F(x) \,
         \mathrm{d}x =  \sum_{k=-n}^n c_k  2 \uppi \delta_{k, \ell} .
       \end{equation}
   Simplify the sum
   \begin{equation}
       \int_0^{2 \uppi} \euler^{-\imag  \ell x} F(x) \,
         \mathrm{d}x =   2 \uppi c_\ell.
       \end{equation}
  
 \end{frame}
 
 \begin{frame}{What did we just do?}
 We've shown that if 
 \begin{equation}
    F(x) = \sum_{k=-n}^n c_k \euler^{\imag  k x},
\end{equation}
then for all $\ell \in -n \dots n$, we have
\begin{equation}
   c_\ell = \frac{1}{2 \uppi} \int_0^{2 \uppi} \euler^{-\imag  \ell x} F(x) \, \mathrm{d}x
\end{equation}
\begin{handlist}
\item Given the function $F$, the complex numbers $c_{-n}$ through $c_n$ are uniquely determined.
\end{handlist}
 \end{frame}
 
 \begin{frame}{The Fourier coefficients}
 
 For any function $F$ that is integrable on the interval $[0,2 \uppi]$ the numbers $c_k$ defined as
 \begin{equation*}
   c_k = \frac{1}{2 \uppi} \int_0^{2 \uppi} \cis(k x) F(x) \, \mathrm{d} x
 \end{equation*}
 are the \emph{Fourier coefficients} of the function $F$.  Generally, $k \in \integers$.
 
 
 \end{frame}
 
 \begin{frame}{I know what you are thinking}
 
 Unless $F$ is fairly simple, we have no chance at finding a formula for the numbers $c_{-n}$ through $c_n$.  
 
 Have no fear: we have
 a tool for that.  Let's use the right-point rule integration to find approximate values for the Fourier coefficients. Using $n$ equal length 
 subintervals of $[0, 2 \uppi]$, we have
\begin{align*}
  c_\ell &\approx \frac{1}{2 \uppi}  \frac{2 \uppi}{n} \sum_{k=0}^{n-1}  \euler^{-  \imag \frac{2 \uppi}{n}  \ell  k} F(x_k), \\
            &=  \frac{1}{n} \sum_{k=0}^{n-1}  \euler^{-  \imag \frac{2 \uppi}{n}  \ell  k} F \left (\frac{2 \uppi}{n} k \right).
\end{align*}
 
  \end{frame}
  
  \begin{frame}{Caveat}
  
  For large values of the integer $\ell$, the integrand of
  \begin{equation*}
   c_\ell = \frac{1}{2 \uppi} \int_0^{2 \uppi} \euler^{-\imag  \ell x} F(x) \, \mathrm{d}x
\end{equation*}
becomes more and more ``wiggly.''  So we should expect that our right point rule quadrature rule will give less and less 
accurate results for larger and larger $\ell$.
  
  \vfill
  
  
  \end{frame}
  
  \begin{frame}{Nice coincidence}
  
  You might be thinking ``Why the low accuracy right point rule?'' Why not something like the trapezoidal rule?  Ha!
  
  For a function $F$ with period $2 \uppi$, the right-point, left-point, and trapezoidal rule all give the same value for
  
  \begin{equation}
     \int_0^{2 \uppi} \cis(k x) F(x) \, \mathrm{d} x
  \end{equation}
  where $k \in \integers$.
  
  \end{frame}
  
  \begin{frame}{DFT defined}
  \begin{define}  Let $a_0, a_1, \dots a_{n-1}$ be numbers (either real or complex).   For every $\ell \in 0 \dots, n-1$, define
  \begin{equation*}
      \widehat a_\ell = \sum_{k=0}^{n-1}  \euler^{-  \imag \frac{2 \uppi}{n}  \ell  k} a_k
  \end{equation*}
  The list of numbers  $\widehat a_0,  \widehat a_1, \dots, \widehat a_{n-1}$ is the \emph{discrete Fourier transform} (DFT) of 
  $a_0, a_1, \dots, a_{n-1}$.
  
  \end{define}
  \end{frame}
  
  \begin{frame}{Doubly indexed things are matrices}
 
 For any $n \in \integers_{>0}$, define $\mathcal{F}_{\ell, k}  = \euler^{-  \imag \frac{2 \uppi}{n}  \ell  k}$. We  have
   \begin{equation*}
      \widehat a_\ell = \sum_{k=0}^{n-1}  \mathcal{F}_{\ell, k} a_k
  \end{equation*}
  we see that the DFT is matrix multiplication. Arranging $\widehat a_0, \dots, \widehat a_{n-1}$ and  
   $a_0, \dots, \widehat a_{n-1}$ as column vectors $\mathbf{a}$ and  $\widehat{\mathbf{a}}$,  respectively, we have
    \begin{equation*}
      \widehat{\mathbf{a}}  = \mathcal{F} \, \mathbf{a}
  \end{equation*}
  
    \end{frame}
    
     \begin{frame}{Magic properties}
     
     \begin{handlist}
     \item When the size of a matrix $\mathcal{F}$ is $n\times n$, ordinarily  the effort required to do the matrix product $ \mathcal{F} \, \mathbf{a}$ is proportional to $n^2$.  
     
     \item But the matrix $\mathcal{F}$ has some special (almost magic) properties that 
     makes the effort only proportional to 
     $n \log_{10}( n)$.
     
     \item The algorithm that utilizes these properties to perform the multiplication quickly is the \emph{Fast Fourier Transform} (FFT).
     
     \item The FFT isn't really a transform, but an algorithm that computes a transform.
     
     \item When $n$ is large, $n^2 \gg n \log_{10}(n)$.
 
 \end{handlist}
  
    \end{frame}
    
  \begin{frame}{Inverses}
  
  For a Fourier matrix $\mathcal{F}$ of size $n\times n$, we have
  \begin{equation*} 
        \overline{\mathcal{F}} \mathcal{F} = n \mathrm{I},
  \end{equation*}
  where $\mathrm{I}$ is the $n \times n$ identity matrix.  Thus
      \begin{equation*} 
        n \mathcal{F}^{-1} = \overline{\mathcal{F}}.
  \end{equation*}
  
  
  \end{frame}
    
    \begin{frame}{What does chatGPT have to say?}
    
    \textbf{Question} Would this document make for a good lecture for 3rd year computer science students?
    
    \textbf{Answer} Yes, this document could serve as a good lecture for 3rd year computer science students, particularly those interested in topics related to mathematics, signal processing, or algorithms.  (Specifics deleted).
    
    \end{frame}
\end{document}